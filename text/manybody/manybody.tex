%\footnotesize\begin{verbatim}
%\end{verbatim}\normalsize

% pw on ncorps:  #thmhcb

\nonstopmode			% prevent exiting when errors occur


\documentclass[headsepline,normalheadings]{book}

\setlength{\textwidth}{6.5in}
\setlength{\oddsidemargin}{0.0in}
\setlength{\evensidemargin}{0.0in}
%\setlength{\textwidth}{7.0in}

\setlength{\parindent}{0pt}
\setlength{\parskip}{2.5mm}

% graphicx  ??

% epsfig doesn't support PDFLatex?
\usepackage{epsfig}
% does this work in pdflatex
% \usepackage{psfig}


\usepackage{html}
\usepackage{nemo}
% \usepackage[english]{babel}
% \usepackage{ahpage}

\title{ 
  {\Huge MANYBODY:} \\
  {\LARGE an accompanying software guide to }\\
  {\bf  Advanced School and Workshop on Computational Gravitational Dynamics } \\
  { Leiden, May 3-13, 2010} \\
  {\small and previously} \\
  {\bf  Escuela Internacional de Simulaciones Galacticas Y Cosmol\'ogicas de N--cuerpos} \\
  {\bf  International School on Galactic and Cosmological N--body Simulations} \\
  { INAOE, Tonantzintla, \\
    Puebla, Mexico, 23 July - 4 August 2006}\\
%  {\small and previously} \\
  {\bf  Zomer school over direkte N--deeltjes simulaties} \\
  {\bf  Summer school on direct N--body simulations} \\
  { Astronomical Institute {\it 'Anton Pannekoek'}, \\
    Amsterdam, Netherlands, 24-30 July 2005}\\
%   {\small and} \\
  { \bf \'Ecole de Dynamique Numerique Ncorps }\\
  { \bf School on Numerical N--body Dynamics }\\
  { Observatoire astronomique de Strasbourg, France, \\
    Strasbourg, 19-22 March 2004}
}

\author{
 {Peter Teuben } \\
% my spiritual co-authors :
% Douglas Heggie - Piet Hut - Jun Makino 2004/2005
% McMillan/Portegies Zwart 2005
% Aguilar/Athanassoula/Dehnen/ 2006
% ???  2010
{

\date{{\small Version 2.0a} \\
      {\small Spring 2010} \\
      {\small Last document revised: \today} \\ 
      {\small \tt URL: http://www.astro.umd.edu/nemo/manybody/} \\
}

%%%%%%%%%%%%%%%%%%%%%%%%%%%%%%%%%%%%%%%%%%%%%%%%%%%%%%%%%%%%%%%%%%%%%%%%%%%
\def\eps@scaling{1.0}%
\newcommand\epsscale[1]{\gdef\eps@scaling{#1}}%
\newcommand\plotone[1]{%
 \typeout{Plotone included the file #1}
 \centering
 \leavevmode
 \includegraphics[width={\eps@scaling\columnwidth}]{#1}%
}%

\newcommand\plottwo[2]{{%
 \typeout{Plottwo included the files #1 #2}
 \centering
 \leavevmode
 \columnwidth=.45\columnwidth
 \includegraphics[width={\eps@scaling\columnwidth}]{#1}%
 \hfil
 \includegraphics[width={\eps@scaling\columnwidth}]{#2}%
}}%


%%%%%%%%%%%%%%%%%%%%%%%%%%%%%%%%%%%%%%%%%%%%%%%%%%%%%%%%%%%%%%%%%%%%%%%%%%%
\begin{document}
\pagenumbering{roman}

%%%%%%%%%%%%%%%%%%%%%%%%%%%%%%%%%%%%%%%%%%%%%%%%%%%%%%%%%%%%%%%%%%%%%%%%%%%
% First page : title, author and date
\maketitle

\begin{htmlonly}
\htmladdnormallink{postscript version}{ftp://ftp.astro.umd.edu/pub/nemo/blabla.ps.gz}
is available. 

{\bf Warning:} The contents of this document are likely to change.
It is advisable not to use links to any pages other than the first
page (this page). Small latex2html conversion problems remain.
\end{htmlonly}

%
%\begin{abstract} 
%
%We have labeled this {\bf ManyBody}  since it is a portal
%to N-body codes (see {\tt manybody.org}) and a convenient placeholder
%for the codes we discuss in this class.
%
%\end{abstract}


%%%%%%%%%%%%%%%%%%%%%%%%%%%%%%%%%%%%%%%%%%%%%%%%%%%%%%%%%%%%%%%%%%%%%%%%%%%
\pagestyle{empty}
%\begin{copyrightpage}

\newpage

% 	Usual preamble with Table of Contents etc.
%\cleardoublepage
\pagestyle{headings}
\addcontentsline{toc}{chapter}{Table of Contents}
\tableofcontents
%%%%%%%%%%%%%%%%%%%%%%%%%%%%%%%%%%%%%%%%%%%%%%%%%%%%%%%%%%%%%%%%%%%%%%%%%%%
%\newpage
\addcontentsline{toc}{chapter}{List of Tables}
\listoftables
%%%%%%%%%%%%%%%%%%%%%%%%%%%%%%%%%%%%%%%%%%%%%%%%%%%%%%%%%%%%%%%%%%%%%%%%%%%
%\newpage
\addcontentsline{toc}{chapter}{List of Figures}
\listoffigures
%%%%%%%%%%%%%%%%%%%%%%%%%%%%%%%%%%%%%%%%%%%%%%%%%%%%%%%%%%%%%%%%%%%%%%%%%%%
% The real start of the manual
%\cleardoublepage
\pagenumbering{arabic}

\chapter*{Preface}

What lies in front of you is an accompanying guide to some of 
the software you might find useful this school.
It is still very much work in progress, with many sparse and
unfinished sections.

%% == this is modest specific

A large number of programs are within the NEMO, Starlab and now
AMUSE packages, 
though a number of useful programs and smaller packages
that are available elsewhere have been assembled here
and made available via the {\it manybody} environment.
See also Appendix B for a more detailed description of this
{\it manybody} environment.

One word of caution though: this Guide is {\bf not} a coherent
story that you should necessarily read front to back, it is more encyclopedic
and will hopefully give you an impression of the capabilities
of this software in case you may need it during this week or in
your subsequent research. On most topics in this guide you can find
much more in-depth information in other manuals and of course 
the source code (needless to say, we would like to advocate some
type of open source policy for our codes).

%
%Something about NEMO/Starlab?  Something about not a coherent story, but
%more some annotated examples? Emphasize on open source code, though
%mention some codes also exist in the closet if you know the right
%people (e.g. pkdgrav)

\section*{Unix Shell}

The {\it manybody} guide assumes some knowledge
of Unix. This means Linux, Solaris and MacOSX should be fine, 
but although cygwin on MS-Windows
gets fairly close, your editor couldn't stomach some of 
its limitations\footnote{YMMV, as they say in this business}.
Most examples show simple snippets
of C-shell command lines; their prompt (e.g. \verb+2%+) 
contains a counter which is normally started at 1 for each section
for easy reference.
There are only a few minor places where these {\tt csh} and 
{\tt sh} examples would differ. We just remind you of
the most important ones:

\begin{enumerate}
\item 
Redirect and merge stderr and stdout into one file 
(see also NEMO's {\tt redir} program):
\footnotesize\begin{verbatim}
  sh:  p1 > log 2>&1
 csh:  p1 >& log
\end{verbatim}\normalsize
\item
Redirect stdout and stderr into separate files:
\footnotesize\begin{verbatim}
  sh:  p1 > out 2>err
 csh:  (p1 > out) >& err
\end{verbatim}\normalsize
\item
Pipe both stdout and stderr into the next program:
\footnotesize\begin{verbatim}
  sh:  p1 2>&1 | p2
 csh:  p1 |& p2
\end{verbatim}\normalsize
\end{enumerate}

Also noteworthy is that 
in most places where a random seed is used (e.g. the NEMO {\tt seed=0} keyword
would take a new seed based on the current date and time) 
a fixed seed is taken (usually {\tt seed=123}) 
so the example can actually be exactly reproduced sans compiler and optimization
differences. In ``real life'' you may not always want to do this!

% yes, intel vs. gcc ; solaris vs. linux ; -g vs. -O on gnu. etc.

\section*{Python shell}

\section*{Examples}

You will find shell script examples in the manual, at the top of the example
you should find a line like

\footnotesize\begin{verbatim}
   # File: examples/ex1
\end{verbatim}\normalsize

which means the example can be found in the directory

\footnotesize\begin{verbatim}
   $NEMO/usr/manybody/examples/ex1
\end{verbatim}\normalsize  % $


\section*{Getting More Help}

After your account has been setup to use the {\it manybody} environment
(see also Appendix A), there are several ways to get more help besides
reading this document:
\begin{itemize}
\item
For most programs the {\tt -h} and/or the {\tt --help} 
command line argument will give a short
reminder of the options and their defaults. This is true for NEMO as well
as Starlab programs.
For NEMO programs the option {\tt help=h} will also give much more extensive help
on the keywords.

\item
The Unix {\tt man} and {\tt gman} commands can be used to view manual pages
for all NEMO programs. Notably the {\it index(1)} and {\it programs(8)}
manual pages given an overview of most commands.

\item
For those packages that have html formatted information, they are
linked in {\tt manybody/index.html}, whereever this may reside on your
system.

\item
Buy your friendly teachers a Coffee or Beer

\end{itemize}

\section*{Acknowledgements}

I would like to thank my co-authors in spirit, 
Douglas Heggie, Piet Hut, and Junichiro Makino for their contributions, 
Simon Portegies Zwart and Christian Boily for the MODEST schools in Amsterdam 
and Strasbourg respectively, where much of this material was first
written down.
In the true spirit of open source, 
numerous authors have contributed their software, which you can find on
the DVD. Without their generous contributions this guide would not be
half the size and a quarter of the contents.
Hopefully we will be able to instill this spirit on the reader too.



\chapter                {Introduction}

Although we will primarely cover direct N-body integrators in this school,
we should recall several {\tt types} are commonly in use
in astrophysics:

\begin{enumerate}
\item
Self-consistent direct N-body integrations\footnote{See Table~\ref{t:codes}}
\begin{equation}
    \ddot{\vec{\bf r}} = -G \sum{ {m\over{\Delta r^3}} \Delta \vec{\bf r}}
\end{equation}

\item
Orbit integrator in a general Gravitational Potential\footnote{{\tt potcode} in NEMO}
\begin{equation}
    \ddot{\vec{\bf r}} = -\nabla \Phi(\vec{\bf r})
\end{equation} 

\item
Orbit integrator in a Flow Potential\footnote{{\tt flowcode} in NEMO}
\begin{equation}
    \dot{\vec{\bf r}} = \vec{\bf v}(\vec{\bf r})
\end{equation}

% \item
% Molecular Dynamics

%% G<0: glass
%% eps^2 < 0,  Pseudo-Newtonion   1/(r-eps)

\end{enumerate}

Another way to divide up the field is by methodologies. There are arguably
three ways to solve the N-body problem: direct Particle-Particle (PP),
Particle-Mesh (PM) and Smooth Field Particle (SFP), also sometimes 
referred to as Self Consistent Field (SCF) method. Various hybrid schemes
also exist, e.g. P$^3$M.
\smallskip
% {\it Q: MC? Fluid Dynamics? Numerical Action?}

%%%%%%%%%%%%%%%%%%%%%%%%%%%%%%%%%%%%%%%%%%%%%%%%%%%%%%%%%%%%%%%%%%%%%%%%%%%
\chapter                {Integrators}

We have made a few N-body integrators available for this school, most 
are summarized
in Table~\ref{t:codes}. A brief description of all of them will follow, and 
for some a more detailed example will follow how they
can be used in the {\it manybody} environment. 
% For data interchange , see chapter 4.


% By nature, these will be ``open source'' products.
% there are some notable examples of integrators
% that are not easily available.% \footnote{gasoline, pkdgrav}


\begin{center}
\begin{table}[h]
\caption{A few good N-body codes}
\begin{tabular}{||l|l|l|l||}

\hline 
{\it code} & {\it method} & {\it author(s)/version}  & {\it data} \\
       &&& {\it format} \\
\hline &&& \\

{\tt firstn} & PP & von Hoerner (1960; via Aarseth)  & \\

{\tt nbody0} & PP & Aarseth (Binney \& Tremaine) & s \\

{\tt nbody\{1,2,4\}} & PP & Aarseth (w/ NEMO interface) & b/s\\

{\tt nbody\{6\}} & PP &  Aarseth  & b \\

{\tt fewbody} & PP & Fregeau   & d \\

{\tt hnbody}  & PP & Rauch \& Hamilton 2004 & \\

{\tt kira} & PP & Starlab team & d \\

{\tt treecode} & PP & Barnes \& Hut 1986, Hernquist 1989 & s \\

{\tt gyrfalcON} & PP & Dehnen 2000 & s \\

% kawaii's code

{\tt gadget} & PP & Springel 2001, 2005  & g \\

{\tt scfm} & SFP & Hernquist 1992 & 205 \\

{\tt quadcode} & SFP & Barnes \& White 1986 & s \\

{\tt CGS} & SFP &  Trenti et al. 2005 & \\

{\tt galaxy} & PM & Sellwood  & b \\

{\tt superbox} & PM &  Fellhauer et al. 2000 & \\

{\tt TPM} & PM &  Bode & \\

{\tt partree} & PP &  Dubinski & \\


\hline 

\end{tabular}
\label{t:codes}
\end{table}
\end{center}

\section{firstn}

The first N-body code (skipping the pioneering analog experiments by 
Holmberg (1941)\footnote{see also {\tt mkh41}}
%Holmberg, Ap.J. 94, 385, 1941)
was arguably written by von Hoerner (1960)  % (Z.f.Astrophys. 50, 184, 1960) 
and has recently
been resurrected by Aarseth and is also available as {\tt firstn} in NEMO. Here is an
example running the code for two crossing times on a self-generated 16-body
Plummer sphere configuration

\footnotesize\begin{verbatim}
   # File: examples/ex1
   # sample input parameter file
1% cat $NEMO/usr/aarseth/firstn/input
16 12.0 0.0 1.0 2.0
1.0 1.0 1.0 0
0.5 0

   # integrate 
2% firstn < $NEMO/usr/aarseth/firstn/input
     N =   16  ETA =  12.0  EPS =  0.000
 
 T =   0.0  Q = 0.38  STEPS =      0  DE =  0.000000  E = -0.357554  TC =   0.0
 ERRORS    RCM VCM DE/E DZ     0.00E+00  0.00E+00  0.00E+00  0.00E+00
 
 T =   2.3  Q = 0.55  STEPS =   1536  DE =  0.000000  E = -0.357554  TC =   1.4
 ERRORS    RCM VCM DE/E DZ     3.17E-16  7.10E-16  2.38E-07  2.97E-08
 
 T =   4.5  Q = 0.53  STEPS =   3624  DE =  0.000001  E = -0.357554  TC =   2.7
 ERRORS    RCM VCM DE/E DZ     2.26E-16  8.86E-16  7.73E-07  2.99E-08

   # get some online help on this program
3% man firstn
 
\end{verbatim}\normalsize

It simply prints out the time T, the virial ratio Q (Q=T/W, thus 0.5 means virial
equilibrium), number of STEPS taken, energy conservation and total energy, and another 
line on the center of mass motion. No actual snapshots are stored in this 
simulation, although the code layout is very similar to that of {\tt nbody0},
and could be adapted quite easily (see below)

You can find the source code in {\tt \$NEMO/usr/aarseth/firstn/}. For 
an explanation of the input parameter file, see the Unix manual
page for this program.
% {\it Q: compare with directcode?}

\section{nbody0}

A basic implementation of the variable timestep integrator (an essential ingredient
to collisional stellar dynamics) was published in Appendix 4B in
Binney \& Tremaine (1987). 
You can find this implementation in Fortran ({\tt nbody0})
as well as an identical C
({\tt nbody00}) in NEMO, with both a simple ASCII dataformat,
as well as the NEMO {\it snapshot} format. Aarseth likes
to refer to this version as the ``Mickey Mouse'' version, and really prefers
you to use the full version, {\tt nbody1}, or better yet,
{\tt nbody4} or {\tt nbody6} (see below).


\footnotesize\begin{verbatim}
   # File: examples/ex2
   # reminder to the valid command line parameters and defaults
1% mkplummer help=
mkplummer out=??? nbody=??? mfrac=0.999 rfrac=22.8042468 seed=0 time=0.0 zerocm=t 
  scale=-1 quiet=0 massname= massexpr=pow(m,p) masspars=p,0.0 massrange=1,1 headline= VERSION=2.6a

   # generate a Plummer sphere with 256 bodies, but with reproducable data
2% mkplummer p256.in 256 seed=123

   # what does nbody00 do?
3% nbody00 help=
nbody00 in=??? out=??? eta=0.02 deltat=0.25 tcrit=2 eps=0.05 reset=t f3dot=f options= VERSION=2.0a

   # some more extensive inline help on the keywords
4% nbody00 help=h
in               : Input (snapshot) file [???]
out              : Output (snapshot) file [???]
eta              : Accuracy parameter - determines timesteps [0.02]
deltat           : When to dump major output [0.25]
tcrit            : When to stop integrating [2]
eps              : Softening length [0.05]
reset            : Reset timestep after datadump (debug) (t|f) [t]
f3dot            : Use more advanced timestep determination criterion? [f]
options          : Optional output of 'step' into AUX []
VERSION          : 13-mar-04 PJT [2.0a]

   # finally, integrate this for about 2 crossing times
5% nbody00 p256.in p256.out
time = 0   steps = 0   energy = -0.257297 cpu =    0.00117 min
time = 0.25   steps = 2519   energy = -0.257303 cpu =      0.006 min
time = 0.5   steps = 5428   energy = -0.257306 cpu =     0.0118 min
time = 0.75   steps = 8207   energy = -0.2573 cpu =     0.0173 min
time = 1   steps = 10914   energy = -0.257304 cpu =     0.0228 min
time = 1.25   steps = 13543   energy = -0.25731 cpu =     0.0282 min
time = 1.5   steps = 16461   energy = -0.257311 cpu =      0.034 min
time = 1.75   steps = 19345   energy = -0.257315 cpu =     0.0398 min
time = 2   steps = 22265   energy = -0.257313 cpu =     0.0462 min
Time spent in searching for next advancement: 0.1
Energy conservation: -1.59973e-05 / -0.257297 = 6.21707e-05
Time resets needed 0 times / 9 dumps


\end{verbatim}\normalsize

Notice that when running {\tt mkplummer} the first two parameters ({\tt in=} and {\tt nbody=})
were not explicitly named, because they were given in the order known to the program. This is a feature of the NEMO command line interface.


\section{nbody1, nbody2}

These two original Aarseth codes are both available with a NEMO wrapper program
({\tt runbody1} and {\tt runbody2)}. It will run the compiled Aarseth fortran 
code ({\tt nbody1} and {\tt nbody2)})
in its own run directory and 
optionally produce {\it snapshot} files for further
analysis. {\tt nbody2}  is much like {\tt nbody1}, except it
treats forces with the Ahmad-Cohen neighbor scheme (see Ahmed Cohen 1973).

\footnotesize\begin{verbatim}

  # File: examples/ex3
  1% mkplummer p100 100 seed=123
                                          # example is now wrong, units are code (virial) now  2(sqrt(2))
  2% runbody1 in=p100 outdir=run1 

       N  NRAND    ETA   DELTAT   TCRIT   QE        EPS

     100      0   0.020     0.2     2.0   2.0E-05   5.0E-02
 
 
       OPTIONS      1   2   3   4   5   6   7   8   9  10  11  12  13  14  15

                    1   2   0   2   0   1   0   0   0   0   1   0   0   0   0
 
       SCALING:   SX  =  0.53  E = -4.69E-01  M(1) = 1.00E-02  M(N) = 1.00E-02  <M> = 1.00E-02
 

       SCALING PARAMETERS:   R* = 1.00E+00  M* = 1.00E+02  V* = 6.56E-01  T* = 1.49E+00
 
 
 T =   0.0  Q = 0.00  STEPS =      0  DE =  0.000000  E = -0.251037  TC =   0.0
 
 <R> =  1.99  RCM =  0.0000  VCM =  0.0000  AZ =   0.00000  T6 =   0  NRUN =  1
 
   BINARY   71  92  0.010  0.010   0.0  0.2751    1.0  0.5503  1.10  1.000    0
   BINARY   87  92  0.010  0.010  -0.1  0.0756    6.8  0.1512  1.32  1.000    0
...
 
 T =   5.7  Q = 0.57  STEPS =  37920  DE = -0.000003  E = -0.251050  TC =   2.0
 
 <R> =  0.85  RCM =  0.0000  VCM =  0.0000  AZ =  -0.00001  T6 =   8  NRUN =  1
 
 
         END RUN   TIME =    5.66  CPUTOT =  0.01  ERRTOT = -0.00002

\end{verbatim}\normalsize


Notice that at T=0 two binaries were already found, in fact star 92 was
involved in both binaries.

% but also notice the bug that Q=0 was set by the wrapper....

% 
% {\tt nbody1} already takes 35 input parameter, of which 15 are in the control
% array {\tt KZ()}.
% {\tt nbody2} takes 42 input parameter, of which 20 are  {\tt KZ()}.
% 
% nbody1:  2+4+6+kz(15)+3+5 = 12+15+8 27+8 35
% nbody2:    5 7    20
% 

\section{nbody4}

The Aarseth {\tt nbody4} code has been recently made available with a GRAPE
interface (a.k.a. Nbody4). A standalone version  (a.k.a. Brut4) is available
as {\tt nbody4} in NEMO. A NEMO frontend, much like the previous ones, is 
available as {\tt runbody4}. One novel addition to this version of {\tt nbody4} is the
capability of running one of several stellar evolution codes
(e.g. Eggleton, Tout \& Hurley, Chernoff--Weinberg). An extensive manual
has recently been made available on how to run it (Aarseth, 2006)}.
Currently the only other
code capable of doing combined dynamical and stellar evolution
is the {\tt kira} code in Starlab.  A very newe project,
{\tt muse}\footnote{{\tt http://www.science.uva.nl/sites/modesta/wiki/index.php/Main\_Page}},
aims to modularize this approach. This is now superseded by 
the AMUSE project\footnote{{\tt http://www.amusecode.org}},

\section{nbody6}

This code is available as is, and with minor modifications could be made to run
via clones of {\tt runbody4}. Fairly extensive documentation is available on how
to modify and run the code.

{\tt nbody6} is available in {\tt \$NEMO/usr/aarseth/nbody6}, 
There is also a recent manual (see Aarseth 2004). Running {\tt nbody6}
is very similar to {\tt nbody4}


\section{nbody6++}

Spurzem (see Khalisi \& Spurzem 2003) published an MPI enabled version
of NBODY6, dubbed {\tt nbody6++}\footnote{and no, it is not rewritten in C++}.
The input parameter list (including the well known {\tt KZ()} and {\tt BZ()}
arrays, now counts 83  ({\tt nbody1} counted 35 parameters, {\tt nbody2} has 42) !

\section{fewbody}

Fregeau et al. (2004) released 
{\it fewbody}, a software package for doing
small-N scattering experiments. It handles an arbitrary number of stars,
and understands arbitrarily large hierarchies. For example, if the
outcome of a scattering experiment between two triples is a stable
hierarchical triple of binaries, {\it fewbody} is smart enough to
understand that the system is in that configuration, and terminate the
integration once it is considered stable. {\it fewbody} uses full pairwise
K-S regularization and an 8th order Runge-Kutta Prince-Dormand
integrator to advance the particles' positions, binary trees as its
internal data structures, both to construct the hierarchies, and also
to isolate unperturbed hierarchies from the integrator, and the
Mardling stability criterion to assess the stability of hierarchies at
each level.

{\it fewbody} also comes with an OpenGL vizualizer called {\tt GLStarView},
which can visualize a simulation as it is being computed by directly
piping the data. You can find fewbody  version 0.21 in {\tt \$NEMO/usr/fregeau},
as wel as glstarview. Within NEMO the {\tt glnemo2} visualization tool has
similar capabilities.

There are  4 programs that come with {\it fewbody}:
{\tt binbin}, {\tt binsingle}, {\tt triplebin} and {\tt cluster},
 each of which understand the
{\tt -h} flag if you get lost. You should however read at least
Fregeau et al. (2004) before using the code.

\footnotesize\begin{verbatim}
   # smash a binary and single star into each other
1% binsingle -D 1 > try1

PARAMETERS:
  ks=0  seed=0
  m0=1 MSUN  r0=1 RSUN
  a1=10 AU  e1=0  m10=1 MSUN  m11=1 MSUN  r10=1 RSUN  r11=1 RSUN
  vinf=0.2  b=3.1  tstop=1e+06  tcpustop=3600
  tidaltol=1e-05  abs_acc=1e-09  rel_acc=1e-09  ncount=500  fexp=3
 
UNITS:
  v=v_crit=11.5357 km/s  l=10 AU  t=t_dyn=4.10953 yr
  M=1.5 M_sun  E=3.97022e+45 erg
 
OUTCOME:
  encounter complete:  t=439.506 (1806.16 yr)  nstar=2 nobj=1:  [0 1:2]  (binary)
 
FINAL:
  t_final=439.506 (1806.16 yr)  t_cpu=0.26 s
  L0=0.660373  DeltaL/L0=1.24727e-08  DeltaL=8.2366e-09
  E0=-0.213334  DeltaE/E0=1.5916e-06  DeltaE=-3.39541e-07
  Rmin=0.00092399 (1.98608 RSUN)  Rmin_i=1  Rmin_j=2



   # view the simulation using Fregeau's program
2% glstarview < try1

   # and since it uses Starlab 'dyn' format, xstarplot can be used as well
3% xstarplot < try1

\end{verbatim}\normalsize

% there is a recent AISRP project by Duncan et al. called SWIFT

\section{hnbody}

{\bf HNBody} (Hierarchical N-body, see Rauch \& Hamilton 2004)
is optimized for 
the motion of particles in self-gravitating systems where the total mass
is dominated by a single object; it is based on 
{\it symplectic integration} techniques 
in which two-body Keplerian motion is integrated exactly. 
For comparison, 
Bulirsch-Stoer and Runge-Kutta integrators are also available.
Particles are divided into three basic groups:
HPWs (Heavy Weight Particles), LWPs (Light Weight Particles), and
ZWPs (Zero Weight Particles).

You can find the code
\footnote{HNBody is under very active development, and source code will be 
available for download later this year. The current version, 1.0.3, is only
available in binary form}
and support files in {\tt \$NEMO/usr/hnbody}. The following example illustrates
how to run a simulation with the Sun and the Jovian planets:

\footnotesize\begin{verbatim}
1% hnbody -h
HNBody version 1.0.3 (linux-x86), released 2004/03/12.
See http://janus.astro.umd.edu/HNBody/ for current information.
Relay questions and bug reports to Kevin Rauch <rauch@astro.umd.edu>.
 
Usage:  hnbody [options] [file1.hnb ...]
Options:
 
  -b        Benchmark machine's HNBody performance and exit.
  -h        Display this help message and exit.
  -l LFILE  Log diagnostic output to LFILE instead of standard output.
  -q        Quiet mode--do not produce diagnostic output.
  -r RFILE  Set kill recovery file name to RFILE (default: recover.dat).
              (Specify /dev/null to disable signal handling.)
  -s SFILE  Restart the integration using SaveFile SFILE.
              (The original input files must also be given.)
  -t TCPU   Estimate required CPU time; use TCPU seconds of effort and exit.
              (No output files will be created or modified.)
  -v        Display driver/HNBody version information and exit.

2% hnbody $NEMO/usr/hnbody/input/jovian.hnb
#
# HNBody version 1.0.3 (linux-x86), released 2004/03/12.
# See http://janus.astro.umd.edu/HNBody/ for current information.
# Relay questions and bug reports to Kevin Rauch <rauch@astro.umd.edu>.
#
Integrator = Symplectic
IntegCoord = Jacobi Order2 Drift-Kick
Corrector = True
AngleUnit = rad
LengthUnit = AU
MassUnit = Msun
TimeUnit = d
StepSize = 365.25
Tinitial = 0
M = 1.00000597682
N = 6
NLWPs = 1
InputOrder = Mass x1 x2 x3 v1 v2 v3
HWP = 0.000954786104043042 3.409530427945 3.635870038323 0.03424028779975 -0.005604670182013 0.005524493219597 -2.663981907247e-06
HWP = 0.00028558373315056 6.612079829705 6.386934883415 -0.1361443021015 -0.004180230691977 0.004003576742277 1.672376407859e-05
HWP = 4.37273164545892e-05 11.16769742623 16.04343604329 0.3617849409933 -0.003265538550417 0.002070723353855 -2.176677219661e-05
HWP = 5.17759138448794e-05 -30.17366005485 1.917641287545 -0.1538859339981 -0.0002241622739519 -0.003107271885463 3.583760257057e-05
LWP = 1e-30 -21.381833467487 32.077998611553 2.4924585571843 -0.0017760562614312 -0.0020608701590214 0.00065809506351528
Tfinal = 365250000
OutputInterval = 36525000
OutputFiles = plan%d.dat
OutputOrder = Time SemiMajorAxis Eccentricity Inclination LongAscendNode ArgPeriapse MeanAnomaly
OutputCoord = Bodycentric
OutputDigits = 16
SaveInterval = 36525000
SaveFiles = save.dat
EnergyInterval = 36525000
EnergyFile = energy.dat
#
Starting integration (Tinitial = 0).
There are 6 particles in the system (5 HWPs, 1 LWP, 0 ZWPs).
 
SaveFile save.dat written (Time = 0, Steps = 0).
Energy   errors:  0.0000e+00 ave, 0.0000e+00 rms, 0.0000e+00 max
Momentum errors:  0.0000e+00 ave, 0.0000e+00 rms, 0.0000e+00 max
Estimated CPU time required: 10.2 s.
 
SaveFile save.dat written (Time = 36525000, Steps = 100000).
Energy   errors:  6.4810e-08 ave, 9.1656e-08 rms, 1.2962e-07 max
Momentum errors:  1.3391e-15 ave, 1.8938e-15 rms, 2.6783e-15 max
CPU     time since start:    1.020 s (1.0200e-05 s per step)
Elapsed time since start:    1.098 s (92.9% CPU ave; 92.9% local)
Approx. time   remaining:    9.883 s (9.883 s local)
 
...
 
Integration complete (Tfinal = 365250000, Steps = 1000000).
Energy   errors:  6.0274e-08 ave, 7.5161e-08 rms, 1.2962e-07 max
Momentum errors:  5.3075e-15 ave, 6.8479e-15 rms, 1.4510e-14 max
CPU     time since start:   10.210 s (1.0210e-05 s per step)
Elapsed time since start:   10.686 s (95.5% CPU ave; 0.0% local)

\end{verbatim}\normalsize

You will find 6 files, {\tt plan0.dat .. plan5.dat} representing
the sun, and planets Jupiter through Pluto. HNbody comes with a
set of supermongo\footnote{the {\tt sm} program itself 
is however not publicly
available} macros especially  designed for the
ASCII tables that are produced.

\section{kira}

{\tt kira} is the flagship integrator of the {\it starlab} package. It is also
one of the few codes that has stellar evolution integrated with stellar
dynamics. {\tt kira} can optionally be compiled with the GRAPE libraries,
to run on the GRAPE hardware, which of course speeds up the gravity computations 
substantially.

\footnotesize\begin{verbatim}
    examples/ex4
    # generate a Plummer sphere with 256 bodies, again with reproducable seed
 1% makeplummer -n 100 -s 123 > k100.in
  rscale = 0.979095
  com_pos = 0  0  0

    # integrate to T=100, output steps D=10
    # notice the csh method of splitting stdout and stderr
 3% (kira -t 100 -D 0.2 < k100.in > k100.out) >& k100.log

    # convert to NEMO snapshots
 4% dtos k100.dat < k100.out

    # look at the lagrangian radii
 5% snapmradii k100.dat log=t | tabplot - 1 2:10 0 100 -1 1 line=1,1

    # integrate with nbody0, need to convert the data first
 6% dtos k100.indat < k100.in
 7% nbody00 k100.indat k100.outdat eta=0.01 deltat=0.2 tcrit=100
 8% snapmradii k100.outdat log=t | tabplot - 1 2:10 0 100 -1 1 line=1,1

 9% hackforce k100.dat - |\
      snapcenter - - "-phi*phi*phi" |\ 
      snapmradii - log=t |\
      tabplot - 1 2:10 0 100 -1 1 line=1,1 xlab=Time ylab="log(M(r))"
10% hackforce k100.outdat - |\
      snapcenter - - "-phi*phi*phi" |\
      snapmradii - log=t |\
      tabplot - 1 2:10 0 100 -1 1 line=1,1 xlab=Time ylab="log(M(r))"


\end{verbatim}\normalsize


\begin{figure}[htb]
\plottwo{kira1.ps}{kira2.ps}
\caption[Lagrangian radii for kira and nbody0]
{Lagrangian radii for {\tt kira} (left, 10\%) and {\tt nbody0} (right, 11\%).
Both have been recentered using the potential $-\Phi^3$.
}
\label{f:kira}
\end{figure}


\section{treecode}

We keep several of the ``original'' 
Barnes \& Hut (1986) treecodes in NEMO. 
The code is basically $\mathcal{O}(N \log{N})$, which makes it very fast 
for large N compared to a direct full N-body code.
The earliest version also
includes the curious self-gravity bug which can occur for large opening angles
under peculiar conditions (see Salmon \& Warren 1994).
The original version around which NEMO was originally
developed is still available as
{\tt hackcode1}. In this example we'll set up a head-on collision between
two Plummer spheres:

\footnotesize\begin{verbatim}
   examples/ex5
   # create our usual reproducable plummer sphere
1% mkplummer p1 512 seed=123
2% mkplummer p2 512 seed=456

   # stack them , but offset them in phase space on a collision course
3% snapstack p1 p2 p3 deltar=5,0,0 deltav=-1.5,0,0 

   # integrate it for a little while
4% hackcode1 p3 p3.out freqout=5 tstop=10
 
init_xrandom: seed used 456
 
       nbody        freq         eps         tol
        1024       32.00      0.0500      1.0000
 
        options: mass,phase
 
        tnow       T+U       T/U     nttot     nbavg     ncavg   cputime
       0.000   -0.1236   -0.8953    138503        60        75      0.00
 
                cm pos   -0.0000    0.0000   -0.0000
                cm vel    0.0000    0.0000   -0.0000
...
        tnow       T+U       T/U     nttot     nbavg     ncavg   cputime
      10.000   -0.1184   -0.8230    111709        42        66      0.37
 
                cm pos    0.0036    0.0134   -0.0027
                cm vel    0.0007    0.0015   -0.0011
 
        particle data written

5% snapplot p3.out nxy=3,3 xrange=-5:5 yrange=-5:5 times=0,1,2,3,4,5,6,7,8 nxticks=3 nyticks=3

   # plot up a diagnostics diagram showing center of mass and energy conservation
6% snapdiagplot p3.out
321 diagnostic frames read
Worst fractional energy loss dE/E = (E_t-E_0)/E_0 = 0.0602005 at T = 3.03125

   # show where the center of mass is, notice the out=. (equivalent to /dev/null)
7% snapcenter p3.out  . report=t
-0.000000 0.000000 -0.000000 0.000000 0.000000 -0.000000
-0.000013 0.000002 -0.000012 -0.000073 0.000032 -0.000070
-0.000026 -0.000001 -0.000039 -0.000027 -0.000050 -0.000224
-0.000022 -0.000004 -0.000098 -0.000009 -0.000036 -0.000287
..
0.003339 0.012733 -0.002340 0.000667 0.001810 -0.000922
0.003468 0.013060 -0.002516 0.000708 0.001670 -0.000952
0.003603 0.013355 -0.002703 0.000738 0.001500 -0.001060

\end{verbatim}\normalsize

\begin{figure}[h!]
\plottwo{coll.ps}{diag.ps}
\caption[Collision between two Plummer spheres]
{Collision between two Plummer spheres ({\tt hackcode1}) 
and the diagnostics overview ({\tt snapdiagplot})
}
\label{f:coll}
\end{figure}


Note that this code does not conserve linear momentum, since the
forces are not symmetric and thus do not exactly cancel. Therefore
you will see a drift in the center of mass (CM), as numerically
shown by the output of {\tt snapcenter}. For a spherical system this should
be a random walk.

In addition to {\tt hackcode1}, there is also
{\tt hackcode1\_qp}, which is slightly more expensive (but more accurate) and adds
quadrupole moment corrections to the force calculations. 

\section{gyrfalcON}

Dehnen (2000) introduced a treecode with multipole expansions, which makes
the code perform $\mathcal{O}(N)$, and competitive to the
Greengard \& Rochlin FMM type codes. The code is written in C++ and fully integrated
in NEMO, i.e. it reads and writes {\it snapshot} files and uses the
{\it getparam} user interface. In the next example we'll take the same input condition
as before, the colliding Plummer spheres:


\footnotesize\begin{verbatim}
# --------------------------------------------------------------------------------------------------------------------
# "gyrfalcON p3 p3.outg tstop=10 step=0.2 hmin=5 VERSION=3.0.5I"
#
# run at  Thu Jul 13 00:17:57
#     by  "teuben"
#     on  "nemo"
#     pid  5815
#
#    time       E=T+V        T          V_in        W         -2T/W     |L|      |v_cm|  tree  grav  step  accumulated
# --------------------------------------------------------------------------------------------------------------------
 0.0000     -0.1317041    1.0572     -1.1889     -1.1855     1.7835  0.031519   4.7e-10  0.00  0.01  0.02   0:00:00.02
 0.031250   -0.1316010    1.0586     -1.1902     -1.1867     1.7840  0.031522   1.2e-09  0.00  0.01  0.01   0:00:00.04
 0.062500   -0.1315449    1.0600     -1.1916     -1.1883     1.7841  0.031525   9.2e-10  0.00  0.00  0.00   0:00:00.05
 0.093750   -0.1314929    1.0618     -1.1933     -1.1893     1.7857  0.031526   7.6e-10  0.00  0.01  0.01   0:00:00.07
. . . 
 9.8750     -0.1310892    0.52518    -0.65627    -0.65459    1.6046  0.031257   1.9e-08  0.00  0.00  0.00   0:00:04.78
 9.9062     -0.1310836    0.52443    -0.65551    -0.65404    1.6037  0.031250   1.9e-08  0.00  0.01  0.01   0:00:04.80
 9.9375     -0.1310790    0.52393    -0.65501    -0.65334    1.6039  0.031241   1.9e-08  0.00  0.00  0.00   0:00:04.81
 9.9688     -0.1310589    0.52360    -0.65466    -0.65291    1.6039  0.031234   1.9e-08  0.00  0.01  0.01   0:00:04.83
 10.000     -0.1311007    0.52330    -0.65440    -0.65275    1.6034  0.031228   1.9e-08  0.00  0.00  0.00   0:00:04.84


\end{verbatim}\normalsize

As you can see, Dehnen's code is quite a lot faster. As for accuracy though, we leave
this as an exercise for the interested reader. Notice this code balances all forces
and thus conserves total linear momentum unlike the BH86 treecode.

\section{superbox}

Fellhauer et al. (2000) wrote a particle-mesh code with high resolution
sub-grids and an NGP (nearest grid point) force-calculation scheme
based on the second derivatives of the potential.  It is also an example of 
a code where initial conditions benefit from knowing the details of the
potential calculation routines. In particular, if the initial conditions
are known to be representing a stable configuration (e.g. a Plummer sphere).
See the program {\tt Setup/plummer.f} in the {\tt Superbox} code distribution.

\section{gadget}

Springel (2000) made {\tt gadget} publicly available.
It is primarily a cosmology code, and based on the treecode, but also incoorporates 
SPH particles
and, for cosmological simulations, the Ewald sum technique to create periodic
boundary conditions. The code was originally written in Fortran, but in 2005 re-released
and completely rewritten in C. It runs in a parallelized mode under MPI, even if you 
have a single processor, you will need the infrastructure to use MPI
(e.g. mpich2 and openmpi)
Also,
{\tt gadget2} will typicall be compiled for a specific physical situation. In the
standard distribution you will find 4 different ones: 
{\it cluster}, {\it galaxy}, {\it gassphere} and 
{\it lcmd\_gas}.

\footnotesize\begin{verbatim}
1%  mkdir galaxy

2%  ln -s $MANYBODY/gadget/Gadget2/Gadget2/parameterfiles

3%  ln -s $MANYBODY/gadget/Gadget2/ICs

4%  lamboot

5%  mpirun -np 2 Gadget2.galaxy parameterfiles/galaxy.param

6%  lamwipe

\end{verbatim}\normalsize

WARNING: this benchmark example (60,000 particles in a galaxy-galaxy collision)
 from the Gadget2  source code takes a little
over 2000 steps,
and can easily take an hour on a single 3 GHz Pentium-4 class CPU. 

To use a dataset from NEMO, the data will have to be transformed using the
{\tt snapgadget} and {\tt gadgetsnap} pair of programs:

\footnotesize\begin{verbatim}
    mkdir galaxy
    ln -s $MANYBODY/gadget/Gadget2/Gadget2/parameterfiles
    ln -s $MANYBODY/gadget/Gadget2/ICs

    mkplummer p10k 10000
    snapgadget p10k galaxy.dat 10000,0,0,0,0
    snap

    lamboot
    Gadget2.galaxy parameterfiles/galaxy.param
    mpirun -np 2 Gadget2.galaxy parameterfiles/galaxy.param
    lamwipe

    gadgetsnap ...

\end{verbatim}\normalsize

Note that in most circumstances you will want to check the units of mass, length and velocity
with those listed in the param file, as Gadget expects this.

The versions of Gadget that are pre-compiled are:

\begin{itemize}
\item Gadget2.cluster  
\item Gadget2.galaxy   
\item Gadget2.gassphere
\item Gadget2.lcdm\_gas 
\end{itemize}

\section{quadcode}

% S. White, Ap.J. 274, 53 (1983), 
% Hernquist, L. \& Barnes, J.  Ap.J. 349,  562 (1990). 

An example of an $\mathcal{O}(N)$ code where the potential and forces are derived from
a potential expansion in spherical harmonics  (White 1983, who dubbed it multipole expansion). 
See also Hernquist \& Barnes, (1990). Fully integrated
in NEMO, and very fast. Excellent to study the dynamics of an isolated galaxy,
stability, check orbit theory etc.

\footnotesize\begin{verbatim}

# File: examples/ex7
1% mkplummer p1024.in 1024 seed=1024

2% quadcode p1024.in p1024.out4 tstop=10

       nbody        freq       eps_r       eps_t        mode       tstop
        1024     64.0000      0.0500      0.0700           3       10.00


          tnow         T+U         T/U     cputime
         0.000   -0.247060     -0.4955        0.00

                cm pos    0.0000   -0.0000    0.0000
                cm vel    0.0000   -0.0000   -0.0000

...

          tnow         T+U         T/U     cputime
        10.000   -0.247041     -0.4872        0.04

                cm pos    0.0208   -0.0033   -0.0009
                cm vel    0.0067    0.0007   -0.0004

        particle data written


#  takes about 2.9" on my laptop

3% snapdiagplot p1024.out4
321 diagnostic frames read
Worst fractional energy loss dE/E = (E_t-E_0)/E_0 = -0.000674741 at T = 6.03125

4% snapmradii p1024.out4 | tabplot - 1 2:10 line=1,1

\end{verbatim}\normalsize

% max nbody?? ``make MBODY=....''


\section{scfm}

Hernquist \& Ostriker (1992) deviced a code termed the 
{\it self-consistent field (SCF) method}, implemented as {\tt scfm} by
Hernquist, with a NEMO wrapper program called {\tt runscfm}.

\footnotesize\begin{verbatim}
  
 # File: examples/ex8 
 2% runscfm `pwd`/p1024.in p1024.out2 dtime=1.0/64.0 nsteps=640 noutbod=16
### Warning [runscfm]: Resolving partially matched keyword dt= into dtime=
snapprint /tmp/p1024.in m,x,y,z,vx,vy,vz header=t > SCFBI
m x y z vx vy vz
[reading 1024 bodies at time 0.000000]
[reading 1024 bodies at time 0.250000]
..
[reading 1024 bodies at time 9.750000]
[reading 1024 bodies at time 10.000000]

    # a quick overview of the evolution 
 3% snapplot p1024.out2/scfm.dat nxy=4,4

    # lagrangian radii, check how much the cluster remains in shape
 4% snapmradii p1024.out2/scfm.dat log=t | tabplot - 1 2:10 line=1,1
 

\end{verbatim}\normalsize

This will have created a series of {\tt SNAPnnn} files in the {\tt p1024.out2}
directory, that have been converted to a file {\tt scfm.dat} in the 
run directory.

Note this program does not use dynamic memory, so the program needs to 
be recompiled for more than the default number of particles (10,000 in
the current code, see {\tt src/nbody/evolve/scfm/scf.h}

\section{CGS}

The {\it Collisionless Galactic Simulator} (CGS) N-body code is a spherical
harmonics Particle-Mesh code. The current version is a largely expanded and
improved version from van Albada by Trenti (Trenti 2005). A convenient
NEMO wrapper, called {\tt runCGS} is available:

\footnotesize
\begin{verbatim}
 # File: examples/ex9 :  
 1% runCGS in=p1024.in out=run2 
>>> snapprint p1024.in x,y,z    > run2/initPOS.dat
x y z
>>> snapprint p1024.in vx,vy,vz > run2/initVEL.dat
vx vy vz
line=1024
Using snapshot in=p1024.in with nbody=1024
>>> CGS.exe >& CGS.log
>>> tabtos fort.90 snap.out nbody,time skip,pos,vel,acc,phi; rm fort.90
[reading 1024 bodies at time 0.000000]
[reading 1024 bodies at time 0.100000]
...
[reading 1024 bodies at time 3.900000]
[reading 1024 bodies at time 4.000000]

    # a quick overview of the evolution 
 2% snapplot run2/snap.out nxy=4,4

    # lagrangian radii, but note masses were not present and had to be added
 4% snapmass run2/snap.out - mass=1 | snapmradii - log=t | tabplot - 1 2:10 line=1,1


\end{verbatim}
\normalsize

% there is something a little odd with this simulation. explain.



\section{galaxy}

Sellwood (1997) contributed this 3-dimensional cartesian code to NEMO. Is uses the
Fourier Analysis/Cyclic Redundancy (FACR) method, pioneered by
Hockney (1965, 1970) and James (1977) for galactic dynamics. Running
the program itself has some of the usual fortran 
restrictions, and a NEMO frontend {\tt rungalaxy}
is available to simplify running this code:

\footnotesize
\begin{verbatim}
1% mkplummer p1024.in 1024
2% rungalaxy p1024.in run1
 galaxy V1.3 
 Maximum number of particles (mbuff):  100000
 Maximum gridsize:  33 33 33
 Greens function creation complete
 1024 8 8192 80000
 305 particles outside the grid at the start
 Run (re)started at time  0.
   and will stop after the step at time   1.05000007
 Time-centering velocities
 Starting step 0
 Starting step 1
...

 Starting step 20
 Un-centering velocities
Time=0
Time=0.5
Time=1
3% snapcenter run1/galaxy.snap . report=t
-0.000846 0.005885 -0.023592 0.003350 -0.004093 0.003552 
-0.006440 -0.002968 -0.026719 0.003096 -0.005531 0.000260 
0.009361 -0.004989 -0.012464 0.013971 -0.005660 0.003792 



\end{verbatim}
\normalsize

\subsection{AMUSE}

Integrators in AMUSE are to be described here.

% see admin.h and reset parameter ( mbuff = 100000 )  to get more particles


\begin{figure}[htb]
\plotone{nt.ps}
\caption[Code Performances]
{Code performances - On the horizontal axis log(N), vertically log(CPU-sec/particle/step).
These are the values for a P4/1.6GHz laptop. Seen in this plot
are nbody0, galaxy, gyrfalcON, hackcode1 and xxx}
\label{f:performance}
\end{figure}


\begin{center}
\begin{table}[h]
\caption[Common integrator keywords]
{Common keywords to N-body integrators differ}
\begin{tabular}{||l|l|l|l|l|l||}

\hline 
{\it code} & {\it time-step}  & {\it output-time-step}  & {\it diag-output} & {\it stop-time} & {\it start-time} \\

\hline &&&&& \\

{\tt firstn}         & & & & & \\

{\tt nbody0}         & & & & & \\

{\tt nbodyX}  (SJA)  & [eta] & deltat & - & tcrit & \\

{\tt fewbody}        &  & & & & \\

{\tt hnbody}         &  & & & & \\

{\tt kira}           &  & & & & \\

{\tt treecode} v2.2 LH      & dtime & nout & - & nsteps &  \\
{\tt treecode1} v1.4 JEB    & dtime & dtout & - & tstop &  \\

{\tt gyrfalcON}      & [hmin] & step & logstep & tstop & - \\

{\tt gadget}         &  & & & & \\

{\tt scfm}           & dtime & noutbod & noutlog & nsteps & - \\


{\tt hackcode1}      & freq & freqout & minor\_freqout & tstop & \\
{\tt quadcode}       & freq & freqout & minor\_freqout & tstop & \\


{\tt CGS}            & dt,dtmin,dtmax & freqout & freqdiag & tstop & \\

{\tt galaxy}         & dy & dtout & dtlog & tstop & \\

{\tt superbox}       & & & & & \\

{\tt partree}       & & & & & \\

\hline 

\end{tabular}
\label{t:codekeys}
\end{table}
\end{center}


%%%%%%%%%%%%%%%%%%%%%%%%%%%%%%%%%%%%%%%%%%%%%%%%%%%%%%%%%%%%%%%%%%%%%%%%%%%
\chapter                {Code Testing}

In the field of hydrodynamics rigorous testing using standard
test cases is quite commonplace. However, 
in the field of stellar dynamics this is not the case, largely due to
the chaotic behavior of the N-body problem 
(though
most papers that introduce a new N-body code of course discuss
their accuracy and performance).
Nonetheless, a
number of efforts have been undertaken to compare codes, performance and
accuracy. We mention a few :


\section{Lecar: 25-body problem - 1968}

This is one of the oldest benchmark in N-body dynamics and also a
very difficult one.
11 numerical integrations of a particular
25-body problem were assembled and presented
by Myron Lecar (1968).
This was essentially a cold collapse model, with all equal-mass 
particles initially at rest, and
in fact the positions of the 25th body was left as a (necessary)
exercise for the reader!
Data are available as {\tt \$NEMODAT/iau25.dat}, though we 
cheated and added the 25th (as the last) body!

\section{Pythagoran Problem}

A really fun setup which has been used in a number of studies
(Burrau in 1913 was the first to report on this,
but unlike the real outcome conjectured it would be 
a periodic solution of the 3-body problem)
is 3 particles on a Pythagoran Triangle with sides 3-4-5 
and relative masses 3:4:5 (Szebehely \& Peters, 1967).
See Section~\ref{s:pyth} for more details and the
coverpage for a sample integration.

% add something about Montgomery's figure-8 and other orbits?


\section{Aarseth, Henon \& Wielen - 1974}

Aarseth, Henon \& Wielen (1974) compared the dynamical evolution of a 
Plummer sphere as integrated with an N-body integrator,
the Monte Carlo method and the fluid dynamical approach.
This paper is also well known for its recipe how to create
an N-body realization of a Plummer sphere. You can find examples
of this in quite a few places, e.g.\\
{\tt \$NEMO/src/nbod4/init/mkplummer.c}\\
{\tt \$STARLAB/src/node/dyn/init/makeplummer.C}\\
{\tt \$NEMO/src/nbody/evolve/aarseth/nbody1/source/data.f}\\
{\tt \$NEMO/usr/aarseth/firstn/data.f}
{\tt \$NEMO/usr/Superbox/Setup/plummer.f}

% a figure here?

{\it cf. long term comparisons Makino et al. did on gravothermal oscillations with Grape}

\section{Kang et al. - Cosmological Hydrodynamics - 1994}


%Kang, Hyesung; Ostriker, Jeremiah P.; Cen, Renyue; Ryu, Dongsu;
%Hernquist, Lars; Evrard, August E.; Bryan, Greg L.; Norman, Michael L.

Kang {\it et al. } compared the simulation results of five
of cosmological hydrodynamic codes from $z=20$ to $z=0$. 5 codes
were used: three Eulerian mesh, and two variants of the
Lagrangian SPH method.
Codes: 
TVD (Ryue et al 1993),
PPM (Bryan et al 1993),
COJ (Cen 1992),
TSPH (Hernquist \& Katz, 1989),
PSPH (Evrard 1988).

% {\it A comparison of cosmological hydrodynamic codes}

\section{Frenk  et al. - 1999}

Frenk {\it et al. } took 12 cosmological codes and compared the
formation of a rich cluster of galaxies.
Again using a number of SPH and
Eulerian (various styles) grid simulations were used.
\smallskip
These simulations became known as the {\tt ``Santa Barbara Cluster''}
benchmark.

\section{Heitmann et al. - 2005}

Comparing cosmological N-body simulations. Codes:
MC2 (Habib et al 2004), % not free
FLASH (Fryxell et al 2000),
HOT (Warren \& Salmon 1993) % not free
GADGET (Springel et al 2001),
HYDRA (Couchman et al 1995),
TPM (Xu 1995, Bode et al 2000).

Apart from the Frenk et al. {\it Santa Barbara cluster} comparison, they
also compared two other simulations:

The initial conditions, as well as some results at z=0, are available
on their 
website \footnote{{\tt http://t8web.lanl.gov/people/heitmann/arxiv/}}.

Power et al - 2003 - on errors.


\section{Hozumi \& Hernquist - 1995}

Testing SCF method on either homogenous spheres or Plummer models, with
given initial virial ratio. This also includes the cold collapse. In
particular, they studied the resulting shapes.

\section{Sellwood: Galaxy Dynamics - 1997}

Sellwood (1997), in {\it Galaxy Dynamics by N-body Simulation},
compared 5 different codes and argued that certain codes...
Comparing direct N-body, tree and grid codes. Plummer
sphere. {\it do what? stability?}


\section{Heggie: Kyoto I - 1997}

This starcluster benchmark was 
organized by Heggie, and presented at the IAU GA in Kyoto in August 1997. 10 results
have been compiled.

\section{Heggie: Kyoto II - 2001}

This followup starcluster benchmark was 
presented at IAU Symposium 208 (Tokyo) on 11 July, 2001. 13 results
have been compiled to this moment.

\section{Notes}
\begin{verbatim}

Also:

 - optimal smoothing papers
      Merritt
      Sellwood
      Dehnen
      Zhan      astro-ph/0507237

      Athanassoula?

 - error stuff

      Power at al 2003

 - relaxation:
      Hernquist & Barnes (1990) : 
      Are some N-body algorithms intrinsically less collisional than others?

 - forwards and backwards integration. 
      van Albada  & van Gorkom - (1977)

 - various collisional codes compared: MC, nbody6, kira, grid codes etc.
   under various circumstances:   See some SPZ papers by:

   Takahashi et al. (2002)
   Spinnato et al. (2003)  
   Joshi et al. (2000)
   Portegies Zwart et al. (2004)
   Gualandris et al. (2004)

- NAM = Numerical Action Method. We'll have code from Shaya. Find solution
  with NAM and integrate it back using e.g. Aarseth and see where they 
  come out.
  e.g.  http://arxiv.org/abs/astro-ph/0512405
   

\end{verbatim}

%%%%%%%%%%%%%%%%%%%%%%%%%%%%%%%%%%%%%%%%%%%%%%%%%%%%%%%%%%%%%%%%%%%%%%%%%%%
\chapter                {Data Conversion}

With this large number of integrators it is unavoidable that dataformats
will be different here. We list a few related to the beforementioned
integrations. Note that if your format is not listed
here, one of these is usually close enough to be modified to work quickly.

\begin{center}
\begin{table}[h!]
\caption{N-body data interchange programs in NEMO}
\begin{tabular}{||l|l|l|l|l|l||}

\hline 
{\it format} & {\it origin} & {\it to-NEMO} & {\it from-NEMO} & {\it comments}\\
\hline &&&&\\

{\tt snapshot} & NEMO    &     -       &        -       & \\

{\tt zeno}       &    ZENO   &   zenosnap &     -       &  csf may also work \\

{\tt dyn}      & starlab &     dtos     &     stod      &  \\

{\tt tdyn}     & starlab &     -        &       -       & use  \\

{\tt ``205''}  & treecode &  atos,atos\_sph    &  stoa, stoa\_sph   &   Hernquist \\

{\tt tipsy}    &  tipsy   &    tipsysnap   & snaptipsy     &   be aware of ascii/binary \\

{\tt gadget}    &  gadget  &    gadgetsnap   & snapgadget    &   \\

{\tt u3}        & nbody1,2  &    u3tos       &    -          & be aware of endianism \\

{\tt martel}    &           &  martelsnap    &    -          &  Martel \\
{\tt heller}    &           &  hellersnap    &    -          &  Heller/Shlosman \\
{\tt rv}       &           &  rvsnap       &   snaprv      & Carlberg \\
{\tt rvc}       &           &  rvcsnap       &         &  Couchman \\
{\tt xvp}      &           &  xvpsnap       &         &  Quinn/Balcells \\

{\it ascii-tables}   &   -   &  tabtos    &    snapprint     &   versatile converter\\

{\it idl}         &    -     &     -      &    snapidl        &   binary IDL format  \\
{\it 3dv}         &    -     &     -      &    snapidl        &   for a variety of 3D viewers \\
{\it xyz}         &   NEMO    &     -      &    snapxyz        &  for xyzview \\
{\it specks}      &  partiview &     -      &    snapspecks      &  for partiview \\
{\it ...}       &  partree     &     -      &    -              &  - \\






\hline 




\end{tabular}
\end{table}
\end{center}

\section{Examples}

Some examples will follow here... See also the next chapter for some examples.



%%%%%%%%%%%%%%%%%%%%%%%%%%%%%%%%%%%%%%%%%%%%%%%%%%%%%%%%%%%%%%%%%%%%%%%%%%%
\chapter                {Projects}

\section{Units}

We need a short session on units. See also NEMO's {\it units(5NEMO)} manual page.
Not just the Heggie \& Matthieu (1986) virial units, but also how to scale them to
real objects, like clusters and galaxies. Discuss stellar evolution in kira
and how this has introduced a hybrid simulation where the scale cannot be
changed!

Aarseth uses {\it N-body Units}, i.e.

{\tt AMUSE} attaches units to numbers, e.g. {\tt m = 100.0 | units.MSun},
{\tt r = 1 | units.parsec}.

\section{Initial Conditions}

Although some initial conditions can be generated from basic principles,
others need the
integrator for force calculations to place the initial conditions in a
specific state consistent with the chosen integrator that uses that force. 
The position of the 25th
body in the IAU 25-body problem is a simple 
example of where an on-the-fly calculation
is needed to guarentee the center of mass is numerically at (0,0,0).

Also important is the issue of quiet starts, as extensively
discussed by Sellwood (1987, 1997). Suppression of shot noise.
See also the new wavelet based code by Romeo et al. 2003, where the
equivalent of a factor of 100 speedup is claimed.

White (1996) suggested a novel way to initialize cosmological simulations,
referred to as ``glass'' initial conditions. They can be obtained when
a Poisson sample in an expanding periodic box is evolved with the
sign of gravity reversed until residual forces have dropped to a negligible
value. Gadget-2 supports this as an option.


\section{Relaxation}

Can we do some interesting experiments in 2D where $\tau_{relax} \approx \tau_{crossing}$?


Also, for a Plummer sphere: show that 2T/W != 1, and that Clausius must be used to study
the equilibrium of the system. How much does it 'relax' for given softening?
Does it depend on N ?


\section{Plummer Sphere}

Explain sampling a DF, use rejection technique, and
using the code show how to make a Plummer sphere. See
Aarseth, Henon \& Wielen (1974).

\section{Galaxy Collisions}


Where is the dark matter in the S+S remnants sometimes called Ellipticals?
(cf. Baes' papers about dark matter in E's). How does the shape
of tidal tails depend on the extent of the dark matter.

Also, this is an ideal scenario to study systematic effects of errors
in the treecode before and after the overlap. Check conservation of energy
and properties of the resulting object(s).


\section{Galactic Discs}

\footnotesize\begin{verbatim}
- Galactic Disks:
  - mkexpdisk/mkexphot      - 
           old NEMO program, nice as toy, but no halo or bulge
           can try and slowly/adiabatically add one. show how bad it is

  - galmake: Hernquist 1993      - has some pitfalls. slow. bad for large N
  - mkkd95: Kuijken & Dubinski
  - MaGalie (Boily et al 2001) 
  - Dehnen's ``Galpot''


  Simulations:
     - bar formation - properties, dependance on ICs, random, Q4
     - heating of the disk
     - interactions:
       - M31/MW
       - M51
       - Antenna

  Subsampling: how to generate good initial conditions?

\end{verbatim}\normalsize



\section{Cold Collapse}


Cold Collapse (cf. v Albada 1981, McGlynn?, but also Lecar's 25-body problem).
Discussed in many papers, e.g. Hozumi \& Hernquist ,
Aarseth, Popaloizou, Lin;  Bertin \& Stiavelli; Boily; Theis. Effects
of softening on shape.

Cold collapse calculations can be done from any spherical particle
distribution by setting the velocities to 0. {\tt snapscale vscale=0.0},
or {\tt snapvirial} can be used to scale to a preferred ratio of $|2T/W|$.
Programs like {\tt mkhomsph} have a direct parameter that controls
the initial virial ratio, so no external scaling is needed.



\begin{figure}[htb]
\plottwo{s1xy.ps}{s1xz.ps}
\caption[Cold Collapse of an N=1000 system]
{Positions of an N=1000 cold collapse at T=0, 1.5 (roughly
the first collapse), 4.0 and 10.0 in an XY projection (left) 
and an XZ projection (right). - try also snapplot3 }
\label{f:s1xy}
\end{figure}

\footnotesize\begin{verbatim}


1% mkhomsph s1 1000 rmin=0 rmax=1.2 2t/w=0
2% gyrfalcON s1 s1.out tstop=10 step=0.05 give=m,x,v,p


3% snapplot s1.out times=0,1.5,4,10 nxy=2,2 nxticks=3 nyticks=3 yvar=y yapp=s1xy.ps/vps
4% snapplot s1.out times=0,1.5,4,10 nxy=2,2 nxticks=3 nyticks=3 yvar=z yapp=s1xz.ps/vps

\end{verbatim}\normalsize

As can be seen in the two projections in Figure~\ref{f:s1xy}, the density center
of the collapsed object does not remain at the origin, despite that a 
perfect
momentum-conserving integrator has been used. Why is that?

To study the structure of the resulting object,
we will translate the center of mass to the density center. Since
we have instructed the integrator to also store the potential (already
available to the code during the integration) in the output data stream, which
can now be used to weigh the particles in {\tt snapcenter} with an appropriate
factor (why $\Phi^3$ and not $\Phi^2$ or $\Phi$?). See also Cruz et al. (2002).

\footnotesize\begin{verbatim}
snapmradii s1.out 0.01,0.1:0.9:0.1,0.99 > s1a.mtab
snapcenter s1.out - "-phi*phi*phi" | snapmradii - 0.01,0.1:0.9:0.1,0.99 > s1b.mtab

snapmradii s1.out 0.01,0.1:0.9:0.1,0.99 log=t > s1a.mtab
snapcenter s1.out - "-phi*phi*phi" | snapmradii - 0.01,0.1:0.9:0.1,0.99 log=t > s1b.mtab


tabplot s1a.mtab 1 2:12 0 10 -2 2 nxticks=9 line=1,1 color=2,3,3,3,3,2,3,3,3,3,2
tabplot s1b.mtab 1 2:12 0 10 -2 2 nxticks=9 line=1,1 color=2,3,3,3,3,2,3,3,3,3,2
        xlab=Time "ylab=log(M(r))"

1000: 10?
4000: 37"
16000: 163"

    an interesting observation of this kind of cold collapse is that
    theoretically (N -> \infty) the density would rise arbitrarely high
    at the collapse (the ``big crunch''), thus at some point the
    assumption of a collisionless simulation are violated.
    Not to mention that softening is then not treated correctly,
    since in general Nbody simulations with softening do not take
    the overlap potential into account.

    test this by using a known 2body problem; 
    check with the work of Aladin in the 70s or 80s ?



\end{verbatim}\normalsize

\begin{figure}[htb]
\plottwo{s1a.ps}{s1b.ps}
\caption[Lagrangian Radii for a Cold Collapse]
{Two Lagrangian Radii calculations: one without centering the
snapshot (left), and one with (right). Notice that the vertical axis is
logarithmic. The mass-radii are plotted at
1,10,20,....90,99\% of the mass. Note that between 20\% and 30\% of the
mass is lost in this simulation.}
\end{figure}




\section{Models for a galactic disk}

Kuijken \& Dubinksi (1995) came up with a novel way to make reasonably
self-consistent disk-bulge-halo models. You can find
their code in {\tt \$NEMO/usr/kuijken/GalactICS-exp}, the binaries have
been placed in {\tt \$NEMOBIN}, as well as a wrapper program
{\tt mkkd95} is available to simplify creating such galaxies in the
NEMO style. In addition, {\tt mkkd95} is optimized to create
multiple random realizations.

\footnotesize\begin{verbatim}


GalactICS/Milky_Way/A> 

make galaxy                                     # mergerv disk bulge halo > galaxy
             this will take a while to compute

    D     B    H
A  8000  4000  6000     13sec
B  1000  1000  1000     48sec
C  4000  2000 10000     48sec
D  1000  1000  1000     98sec
tabtos galaxy A0.snap nbody,time m,pos,vel

These are DBH (Disk-Bulge-Halo, in that order in a snapshot) models

snapmstat A0.dat sort=f
0 0:7999  = 8000 Mass= 0.000108822 TotMas= 0.87058 CumMas= 0.87058
1 8000:11999  = 4000 Mass= 0.00010631 TotMas= 0.425242 CumMas= 1.29582
2 12000:17999 = 6000 Mass= 0.000819365 TotMas= 4.91619 CumMas= 6.21201


0=disk   8000
1=bulge  4000
2=halo   6000


snapxyz A0.dat - | xyzview - nfast=18000 maxpoint=18000 scale=16

snapplot A0.dat color='i<8000?2.0/16.0:(i<12000?3.0/16.0:4.0/16.0)'
    colors in yapp are 0..1 and since we use PGPLOT, we only  have
    16 colors.   0/16=b  1/16=white 2=red 3=green 4=blue
e.g. to see the colors

snapplot A0.dat color=x

snapxyz A0.dat - 

snapxyz A0.dat - color='i<8000?1:(i<12000?2:4)' | xyzview - maxpoint=18000 nfast=18000 scale=8 fullscreen=t





snapprint A0.dat  x | tabhist -
minmax about 5
snapprint A0.dat  vx | tabhist -
minmax about 1

thus T=2.pi.R/v = 6*5/1=30 !!!

thus freqout=1/30 for an orbit
thus freq=1/1500 

gyrfalcon:   2^hmin = 1500   ::   hmin=10..11  !!


gyrfalcON A0.dat A1.dat tstop=1/30 step=1/300 hmin=10


 0          -2.47565      0.49177 1.2903     7.2e-09  0.02   0.25       0.27      0.27
 0.00097656 -2.475653     0.49177 1.2903     7.4e-09  0.01   0.28       0.29      0.57
 0.0019531  -2.475656     0.49177 1.2903     7.4e-09  0.03   0.25       0.28      0.85
....

 0.033203   -2.475671     0.4919  1.2903     5.1e-09  0.02   0.26       0.28      9.93
 0.03418    -2.475674     0.4919  1.2903     4.9e-09  0.02   0.26       0.28     10.21

snapdiagplot A1.dat
11 diagnostic frames read
### Warning [snapdiagplot]: Autoscaling time. MinMax=-0.00170898 0.0358887
Worst fractional energy loss dE/E = (E_t-E_0)/E_0 = 9.63054e-06 at T = 0.0341797

time gyrfalcON A0.dat . tstop=1 step=1/30 hmin=10
 0          -2.47565      0.49177 1.2903     7.2e-09  0.02   0.26       0.28      0.28
 0.00097656 -2.475653     0.49177 1.2903     7.4e-09  0.03   0.25       0.28      0.58
...
 0.99902    -2.475605     0.49946 1.2903     1.5e-08  0.02   0.27       0.29   298.529
 1          -2.475602     0.49945 1.2903     1.5e-08  0.02   0.25       0.28   298.809


290.800u 8.070s 6:53.62 72.2%   0+0k 0+0io 376pf+0w

snapdiagplot A1.dat
31 diagnostic frames read
### Warning [snapdiagplot]: Autoscaling time. MinMax=-0.05 1.05
Worst fractional energy loss dE/E = (E_t-E_0)/E_0 = 7.54072e-05 at T = 0.5



---

gyrfalcON A0.dat A2.dat tstop=1 step=1/100 hmin=9

 0          -2.47565      0.49177 1.2903     7.2e-09  0.02   0.25       0.27      0.27
 0.0019531  -2.475656     0.49177 1.2903     7.1e-09  0.02   0.21       0.23      0.51

snapdiagplot A2.dat
101 diagnostic frames read
### Warning [snapdiagplot]: Autoscaling time. MinMax=-0.05 1.05
Worst fractional energy loss dE/E = (E_t-E_0)/E_0 = 8.70601e-05 at T = 0.521484

gyrfalcON A0.dat A3.dat tstop=10 step=1/10 hmin=8
 0          -2.47565      0.49177 1.2903     7.2e-09  0.02   0.26       0.28      0.28
 0.0039062  -2.475652     0.49177 1.2903     7.4e-09  0.02   0.25       0.28      0.56
...
  9.9961     -2.475596     0.49744 1.2903     1.9e-08  0.02   0.26       0.28   744.695
 10         -2.475598     0.49744 1.2903     1.8e-08  0.02   0.26       0.28   744.975
725.720u 19.340s 17:38.84 70.3% 0+0k 0+0io 376pf+0w


snapdiagplot A3.dat
101 diagnostic frames read
Worst fractional energy loss dE/E = (E_t-E_0)/E_0 = 8.21485e-05 at T = 0.5



\end{verbatim}\normalsize

\subsection{Potential}

\footnotesize\begin{verbatim}
time mkkd95 junk1 40000 80000 60000                         125\"
gyrfalcON junk1 junk1.pot give=m,x,v,p tstop=0                3.5\"


foreach n (32 64 128 256 512 1024)
  snapgrid junk1.pot junk1-$n.ccd xrange=-10:10 yrange=-10:10 yvar=z zvar=t zrange=-0.1:0.1 
        mean=t evar=phi nx=$n ny=$n
end

\end{verbatim}\normalsize  %$

\section{Detonating Galaxies: testing the treecode}

\footnotesize\begin{verbatim}

mkplummer pp.in 15000
mkplummer ps.in 4000
snapscale ps.in ps1.in mscale=0.2 rscale=0.2
hackforce ps1.in ps2.in
snapvirial ps2.in ps3.in rscale=f virial=1
gyrfalcON ps3.in ps3.out tstop=10 step=0.2
snapmradii ps3.out | tabplot - 1 2:10

snapshift pp.in pp1.in 0.3,0.3,0.3
snapshift ps3.in ps4.in 7,7,7 -0.25,-0.25,-0.25

snapadd pp1.in,ps4.in  detonate.in


\end{verbatim}\normalsize  

\section{Accuracy of a code}

Use the van Albada and van Gorkum technique to integrate a code
forwards, storing many snapshots. For each snapshot you then integrate
backwards to t=0 (some code allow $dt<0$, others needs to be
rescaled v = -v) and comparing the two snapshots with snapcmp.
Plot the results as function of T. What kind of bahavior do we see?
Liapunov?  Which codes are ``good'', which are bad, and can we 
understand this?


\section{Collisionless?}

Athanassoula - papers on importance of resonances, live bars. What is a good N to do
galactic dynamics. Martin Weinberg now claims $10^7-10^8$. 
Also check the approach Romeo et al. (2003) took using wavelets.

\section{Comparing}

Take two Nbody codes and compare their potential. For example, use Sellwood's 
{\it galaxy} PM/FFT code and Dehnen's {\tt gyrfalcON} TreeCode and Trenti's
(GCS) SFP expansion code,
and compare the
potential for a homogenous sphere of different degrees of flattening.
Another interesting parametrization is that of two spheres at
different separations (collisions of galaxies). See also
Figure~\ref{f:coll}.


%%%%%%%%%%%%%%%%%%%%%%%%%%%%%%%%%%%%%%%%%%%%%%%%%%%%%%%%%%%%%%%%%%%%%%%%%%%
\chapter                {Visualization}

Some brief but wise words about packages we all have some experience with.
Obviously its a big ugly world out there, and there is a lot more under the
sun than this.

\section{NEMO}


\subsection{snap3dv}

The {\tt snap3dv} program in NEMO converts data to a variety of popular
3D viewers, but in addition special converters are available 
for {\tt partiview} ({\tt snapspecks}) and {\tt xyzview} ({\tt snapxyz}).
Even simple programs like {\tt snapprint} can work very effectively.

\subsection{xyzview}

Comes with NEMO, needs the VOGL (Very Ordinary GL) library, which emulates
GL on classic X with no acceleration. Can only plot colored dots, no finite
size points. Poor for presentations. (manual) animations, can also show
the orbit for one selected star from the simulation. 


\footnotesize\begin{verbatim}
   # standard model-A from Kuijken&Dubinski with 8000,4000,6000 in disk-bulge-halo
1% mkkd95 a0.dat

   # display them in Red/Green/Blue using snapplot 
2% snapplot a0.dat color='i<8000?2.0/16.0:(i<12000?3.0/16.0:4.0/16.0)' yvar=z xrange=-8:8 yrange=-8:8

   # display them in Red/Green/Blue using xyzview
3% snapxyz a0.dat - color='i<8000?1:(i<12000?2:4)' | xyzview - maxpoint=18000 nfast=18000 scale=8
\end{verbatim}\normalsize

{\it need an orbit example here as well}

\subsection{glnemo2}

A recent addition from the Marseille group, {\tt glnemo} is a versatile 
interactive 3D viewer that can also display SPH/gas particles. Examples
dataset are in {\tt \$NEMO/src/nbody/glnemo/snapshot}. 
Recent updates to  {\tt glnemo} included making animations. A completely
rewritten version, {\tt glnemo2} is now the default version, and the
previous version is deprecated.

\section{Starlab}
\subsection{xstarplot}

Starlab's visualization program is called {\tt xstarplot}. More to say on this.

\footnotesize\begin{verbatim}
   # create a plummer
1% makeplummer ...

   # integrate it
2% kira ...

   # display it
3% xstarplot ....
\end{verbatim}\normalsize

\section{partiview}

An NCSA product. Excellent interactive capabilities. Uses GL. A special
format, dubbed {\tt tdyn}, can be produced by {\tt kira} in order for
{\tt partiview} to seamlessly zoom in space {\bf and} time. NEMO has
a conversion program {\tt snapspecks}.

\section{tipsy}

This is a neat X based application. Might have problems with non-8 bit 
display.\footnote{See comments in {\tt \$NEMO/src/nbody/evolve/aarseth/triple/README.pjt}}

\footnotesize\begin{verbatim}

1% startx -- :1 -depth 8

2% cd $MANYBODY/tipsy/
3% tipsy

> openascii run99.ascii
> readascii run99.bin
  read time 14.970800
> loadb 14
  used time 14.970800, hope you don't mind master
> xall
> quit
\end{verbatim}\normalsize



\section{Tioga}

Written in ruby, in active development. Good quality PDF output, animations are also possible.

\section{glstarview}

Comes with Fregeau's code. Data from the integration can be directly  piped
(or tee'd) into this program for visualization. Notice that {\tt xstarplot}
in starlab can also do this.

\footnotesize\begin{verbatim}
1% binbin -D 1 | glstarview
\end{verbatim}\normalsize


\section{starsplatter}

StarSplatter is a tool for creating images and animations from
astrophysical particle simulation data. It treats each particle as a
Gaussian "blob", with an exponential fall-off similar to the SPH
smoothing length or gravitational softening length common in
astrophysics simulations. It also properly anti-aliases points, so
even if the particles are very small the results may look better than
one would get by simply binning particle counts to produce an
intensity image.

See {\tt http://www.psc.edu/Packages/StarSplatter\_Home/}

You can also find tools there to help you set up a scene, choose a pan and
zoom around.

\section{gravit}

This program has recently been developed, and has various neat visualizaton
options and animates simulations. It also displays the oct-tree of the
underlying Barnes \& Hut treecode. There are plans to integrate this with
some of the existing codes.

{\tt http://gravit.slowchop.com/}

\section{povray and blender}

Great to create photo-realistic images. Slightly steep learning curve, but
excellents books available. And {\tt blender}. For an example, see
{\tt http://www.bottlenose.demon.co.uk/galactic/index.htm}.

{\tt http://www.povray.org/}
{\tt http://www.blender.org/}

\section{spiegel}

As part of the gravitySimulator project at RIT, visualization software using Java
and Java3D is being developed.

\verb+http://www.cs.rit.edu/~grapecluster/+

\section{visit}

{\tt http://www.llnl.gov/visit/}

\section{ifrit}

{\tt http://home.fnal.gov/~gnedin/IFRIT/}
uses VTK and QT and is
written in C++. It is a powerful
tool that can be used to visualize 3-d data sets.

\section{xnbody}

an online visualization tool for nbody6++, written by
Sonja Habbinga for her diploma thesis at the
Research Centre Juelich in Juelich, Germany. 
{\tt http://www.fz-juelich.de/jsc/JSCPeople/habbinga}


\section{OpenDX}

Open source visualization package, written by IBM. See 
{\tt http://www.opendx.org/}. Can be quite memory intensive, but has
excellent tools for particle as well as grid data.

\section{AstroMD / VisiVO}

{\bf cosmolab} distributes an open-source vizualization package 
called {\tt AstroMD}.  Uses the Visualization Tool Kit ``vtk''
(see {\tt http://public.kitware.com/VTK/}).

\section{xgobi, ggobi}

Great for multi-variate analysis after you have accumulated hundreds
or thousands of simulations. There is also a limited version of
this kind of dynamic query technique available in 
NEMO's program {\tt tabzoom}, which uses pgplot for visualization
and a simple command line interface for interactions.

\section{python}

Excellent language, now very popular. Graphics interface {\tt matplotlib},
or alternatively {\tt qwt}.


\section{R}

Although R is a statistics package, it has a very nice integrated
way for visualization, and allows extending the package by using
your own compiled code. It could potentially be useful for our
type of analysis. URL: {\tt http://cran.us.r-project.org/}

% some claim R is not good for large datasets, and of course loops are slow...

\section{IDL}

Although this program is not freely available, 
many people have written display and 
analysis routines for IDL that are freely available.
Gadget comes with some
routines. See also NEMO's inefficient {\tt snapidl} routine to export data for IDL.
The Berkeley group (Marc Davis) has some routines on their website
and some of your instructors admitted having used it (but they never inhaled).

A public version, GDL,  is available which is largely IDL V6 compatible, and uses
{\tt gnuplot}\index{gnuplot}\index{GDL} are the graphics engine.

% give a snapidl example...

\section{ImageMagic}

You always need image conversion programs. {\tt ImageMagic} is one of 
the better packages for this. Much like NEMO and Starlab, ImageMagic
is a collection of image manipulation programs (e.g. convert, montage, mogrify).
Reads and writes just about any image format, including some flavors
of FITS. Most linux
distributions make this package directly available.


\section{Movies}

Animations are still the most popular way to convey your results
in a presentation. Good tools exist, but the intrusion of closed
sourced formats (or CODECs within a particular format)
has scattered the field.

mpeg, animated gif, avi, mpg2avi ....

\begin{verbatim}
http://bmrc.berkeley.edu/frame/research/mpeg/mpeg2faq.html

http://www.bergen.org/AAST/ComputerAnimation/Help_FAQs.html

http://the-labs.com/GIFMerge/

$NEMO/csh/mkmpeg_movie : some example scripts to make movies

\end{verbatim} 
%$
%%%%%%%%%%%%%%%%%%%%%%%%%%%%%%%%%%%%%%%%%%%%%%%%%%%%%%%%%%%%%%%%%%%%%%%%%%%
\chapter                {Exercises}

\begin{enumerate}

\item 
Find out what time step criterion the first N-body code (von Hoerner)
was using. How do {\tt nbody0} and this program ({\tt firstn})
scale in their performance if you want to reach the same accuracy?

\item
The output of the example {\tt firstn} output listed earlier
(examples/ex1), can depend on the compiler, and its flags. For example
the Q=0.55 with 3624 steps can also become Q=0.65 with 3569 steps!
Can you understand this. Is this worrysome?


% make clean all ex1 FC=gfortran FFLAGS=-g

\item
What are these peculiar ``{\tt time resets}'' quoted in one of the {\tt nbody0}
examples.

\item
Why was the location of the 25th particle not given in the IAU 25-body problem
(Lecar 1968)?

\item
Take {\tt hackcode1}, There are two ways to get more accurate results:
add a quadrupole term (as implemented in {\tt hackcode1\_qp}) or decrease
the critical opening angle ({\tt tol=}). Device an experiment to show
which of the two is the less compute intensive. Does it depend on the 
number of particles or the configuration?

\item
Back to kindergarden: connect the dots in Figure~\ref{f:performance}
and label which codes they are.


\item
Take an original 
integrator, one without a NEMO interface, and write a user
friendly shell script (sh, csh, python, perl, ...) that can take a
NEMO snapshot input, and produces a NEMO snapshot output file, for
given timestep dt, output timetep dt\_out and a final integration
time of tstop.

\item
What is better if you want to find the center of a snapshot.
Using small N and doing a lot of (K) experiments, or 

\item
For a given configuration (for example a Plummer sphere), define a statistical
measure for the accurary of the treecode as a function of the critical
opening angle, and compare the classical BH treecode O(NlogN) with and
without quadrupole corrections to that of the O(N) Dehnen treecode.

\item
Follow up on the previous exercise and add Gadget2 in the comparison.

\item
Use CVS (see Appendix C) and study the performance of gyrfalcON on the following
dates:  now, 1-jul-2006, ..

\item
Try out the {\tt galaxy} screensaver in Linux, study their code
and generate initial conditions for NEMO integrators. How realistic are they?
Compare Toomre \& Toomre's (1972) classic work.

\item
Study the Holmberg (1941) galactic disk, in particular the stability properties.

\end{enumerate}


%%%%%%%%%%%%%%%%%%%%%%%%%%%%%%%%%%%%%%%%%%%%%%%%%%%%%%%%%%%%%%%%%%%%%%%%%%%
\chapter                {References}


Aarseth, S.J. - 
{\it Gravitational N-body Simulations : Tools and Algorithms} 
(Cambridge Univ. Pr., 2003),

Aarseth, S.J. - 2004.. {\it NBODY6 Users Manual} : {\bf man6.ps}

Aarseth, S.J. - 2006.. {\it Introduction to Running Simulations with NBODY4} : 
{\bf nbody4\_intro\_0602.pdf}


Aarseth, S. J.; Henon, M.; Wielen, R. - 1974
{\it A comparison of numerical methods for the study of star cluster 
dynamics} : {\bf 1974A+A....37..183A.pdf}~\footnote{papers marked 
in {\bf boldface} are available electronically on the CD in the {\tt papers/} directory}

van Albada, T. S.; van Gorkom, J. H.	- 1977 -
{\it Experimental Stellar Dynamics for Systems with Axial Symmetry}:
{\bf 1977A+A....54..121V}



Barnes, J. \& Hut, P. - 1986 - Nature - 324, 446.

Barnes \& Hut - 1989
{\it Error analysis of a tree code} - {\bf 1989ApJS...70..389B.pdf}.

Binney, J. \& Merrifield, M. {\it Galactic Astronomy}, Princeton University Press. 1998.

Binney, J. \& Tremaine, S. - {\it Galactic Dynamics}, Princeton University Press. 1987. 

Cruz, F.; Aguilar, L. A.; Carpintero, D. D
{\it A New Method to Find the Potential Center of N-body Systems} - 2002RMxAA..38..225C
{\bf RMxAA..38-2\_cruz.pdf}

Bode, Paul, Ostriker, Jeremiah P., Xu, Guohong
{\it The Tree Particle-Mesh N-Body Gravity Solver} - 
{\bf  2000ApJS..128..561B.pdf}

Dehnen, W. - {\it A Very Fast and Momentum-Conserving Tree Code} - 2000
astro-ph/0003209 : {\bf 0003209.ps}, {\bf 2002JCP...179...27D.pdf}

Fellhauer et al - 2000 
{\it SUPERBOX - An efficient code for collisionless galactic dynamics}
NewA, 5, 305. {\bf 2000NewA....5..305F.pdf} 

Fregeau, J.M., Cheung, P., Portegies Zwart, S.F., and Rasio, F.A. (2004)
{\it Stellar Collisions During Binary-Binary and Binary-Single Star Interactions}
astro-ph/0401004 : {\bf 0401004.ps}

Frenk, C. et al - 1999 
{\it The Santa Barbara Cluster Comparison Project: A Comparison of Cosmological Hydrodynamics Solutions}
{\bf 1999ApJ...525..554F}

Goodman, Jeremy; Heggie, Douglas C.; Hut, Piet - 1993 -
{\it On the Exponential Instability of N-body Systems} :
{\bf 1993ApJ...415..715G.pdf}

Gualandris, A., Portegies Zwart, S., and Tirado-Ramos, A. - 2004
{\it Performance analysis of direct N-body algortithms on highly distributed systems}:
IEEE - {\bf gualandris2004.pdf}

Heggie, D. \& Hut, P. 2003, 
{\it The Gravitational Million-Body Problem}, 
Cambridge University Press.

Heitmann, Katrin; Ricker, Paul M.; Warren, Michael S.; Habib, Salman - 2005
{\it Robustness of Cosmological Simulations. I. Large-Scale Structure}
{\bf 2005ApJS..160...28H.pdf} - or {\bf 0411795.pdf} from astro/ph

Hernquist, L. 1989. {\it Performance characteristics of tree codes} Ap J Supp. 64, 715.
{\bf 1987ApJS...64..715H.pdf}.

Hernquist, L. \& Barnes, J. 1990 - 
{\it Are some N-body algorithms intrinsically less collisional than others?}
Ap.J. 349,  562. {\bf 1990ApJ...349..562H.pdf}

Hernquist \& Ostriker - 1992 - {\it A self-consistent field method for galactic dynamics}
{\bf 1992ApJ...386..375H.pdf}.

Hernquist - 1993 - ApJS - how to create initial conditions (we have the code, but doesn't
seem to work well)

Hockney \& Eastwood - 1970? {\it Simulations using Particles} - there is now a 2nd edition.

Hozumi \& Hernquist - 1995 - ....

Holmberg, E. - 1941 - Ap.J. 94, 385.
{\bf 1941ApJ....94..385H.pdf}.

Hurley et al.- astro-ph/0507239

Hut, P. \& Makino, J., 2003  {\it Moving Stars Around} :
{\bf acs\_v1\_web.pdf}

Hut, P. \& McMillan, S.L, 1987  {\it The Use of Supercomputers in Stellar Dynamics} :

Joshi, Kriten J.; Rasio, Frederic A.; Portegies Zwart, Simon - 2000 -
{\it Monte Carlo Simulations of Globular Cluster Evolution. I. Method and Test Calculations},
ApJ 540, 969: 
{\bf 2000ApJ...540..969J.pdf}
% is this the starlab paper?

Kang, H. et al - 1994
{\it A comparison of cosmological hydrodynamic codes}
{\bf 1994ApJ...430...83K}

Khalisi, E., \& Spurzem, R., 2003 - {\it NBODY6 : Features of the computer code}.
(last update: 2003, May 02)

Kuijken \& Dubinski - 1995 {\it Nearly self-consistent disc-bulge-halo models for galaxies}.
MNRAS, 277, 1341 : {\bf 1995MNRAS\_277\_1341K.pdf}.

James, R. - 1977 -  J.Comp. Phys. 25, 71.

Lecar, M. - {\it The Standard IAU 25-body problem} - 1968, Bull.Astr. 3, 91. 

Steven Phelps, Vincent Desjacques, Adi Nusser, Edward J. Shaya - 2005
{\it Numerical action reconstruction of the dynamical history of dark matter haloes in N-body simulations} -
astro-ph/0512405:   {\bf 0512405.pdf}

Portegies Zwart, Simon F.; Baumgardt, Holger; Hut, Piet; Makino, Junichiro; McMillan, Stephen L. W. -
2004 - {\it Formation of massive black holes through runaway collisions in dense young star clusters}
Nature 428, 724 (astro-ph/0402622): {\bf 0402622.pdf}

Rauch, K. \& Hamilton, D. - 2005 - in preparation. See also
{\tt http://janus.astro.umd.edu/HNBody}


Romeo, Alessandro B.; Horellou, Cathy; Bergh, Jöran - 2003 - 
{\it N-body simulations with two-orders-of-magnitude higher performance using wavelets}
MNRAS, 342, 337 : {\bf 2003MNRAS.342..337R.pdf}

Salmon, J.K. \& Warren, M.S. - 1994 -
{\it Skeletons from the Treecode Closet} J. of Comp. Phys, {\bf 111}, 136:
{\bf skeletons.ps}

Sellwood, J.A. - {\it The art of N-body building} - 1987, Ann.Rev A\&A, 25, 151 : 
{\bf 1987ARA+A..25..151S.pdf}


Sellwood (1997, {\it Computational Astrophysics} :
in {\it Galaxy Dynamics by N-body Simulations},
ed Clarke \& West,  ASP  Conf  series  v123, p215) - astro-ph/9612087
{\bf 9612087.ps}

Spinnato, Piero F.; Fellhauer, Michael; Portegies Zwart, Simon F. - 2003 -
{\it The efficiency of the spiral-in of a black hole to the Galactic Centre} :
MNRAS 244, 22S.: {\bf 2003MNRAS.344...22S.pdf}.

Springel V. - 2005 -
{\it The cosmological simulation code GADGET-2}
{\bf 2005MNRAS.364.1105S.pdf} {\bf gadget2-paper.pdf}

Springel V., Yoshida N., White S. D. M., 2001, New Astronomy, 6, 51. 
{\it GADGET: A code for collisionless and gasdynamical cosmological simulations}:
{\bf gadget-code-paper.pdf}.

Szebehely, Victor, Peters, C. Frederick - 1967 - 
{\it Complete solution of a general problem of three bodies}:
{\bf 1967AJ.....72..876S.pdf}

Takahashi, Koji; Portegies Zwart, Simon F. - 2002 -
{\it The Evolution of Globular Clusters in the Galaxy} - ApJ 535, 759:
{\bf 2000ApJ...535..759T.pdf}

Theis, C. {it  Two-body relaxation in softened potentials} - 1998, Astron.Astroph. 330, 1180:
{\bf theis98.pdf}.

Trenti 2005.

von Hoerner, S. - 1960 - Z.f.Astrophys. 50, 184.
{\bf 1960ZA.....50..184V.pdf}

% von Hoerner, S. - 1963 Z.Astrophys. 57, 47.

Weinberg, M. - 2001a - 
{\it Noise-driven evolution in stellar systems - I. Theory}:
{\bf 2001MNRAS.328..311W}

Weinberg, M. - 2001b - 
{\it Noise-driven evolution in stellar systems - II. A universal halo profile}:
{\bf 2001MNRAS.328..321W}


White,S.D.M. - 1983 - Ap.J. 274, 53 : {\bf 1983ApJ...274...53W.pdf}

White,S.D.M. - 1996 {\it ...DaTitle....},
in: {\it Les Houches Lecture Notes}
{\bf 94100043.ps}

White \& Barnes - 1984 - {\bf 1984MNRAS.211..753B.pdf}

Zhan - astro-ph/0507237 {\bf Zhan-0507237.pdf}

\section*{URLs}

\footnotesize
\begin{verbatim}

http://modesta.science.uva.nl/modest/modest5c/index.html  this N-body school (2005)
http://astro.u-strasbg.fr/scyon/School/                   the previous N-body School (2004)
http://www.artcompsci.org/                                The Art of Computational Science
http://www.manybody.org/                                  Manybody portal (including MODEST)
http://www.astro.umd.edu/nemo                             NEMO website
http://www.ids.ias.edu/~starlab/                          Starlab website
http://muse.li/                                           MUSE
http://www.amusecode.org/                                 AMUSE

\end{verbatim}
\normalsize


%%%%%%%%%%%%%%%%%%%%%%%%%%%%%%%%%%%%%%%%%%%%%%%%%%%%%%%%%%%%%%%%%%%%%%%%%%%
\appendix
%%%%%%%%%%%%%%%%%%%%%%%%%%%%%%%%%%%%%%%%%%%%%%%%%%%%%%%%%%%%%%%%%%%%%%%%%%%
% \cleardoublepage
% \part{Appendices}
\chapter                {Setting Up Your Account}

\section{Leiden 2010 setup}

We will be using Linux workstations, running Suse 10.1 (x86-64) with dual 64bit 
Intel 3 GHz CPU processors (totaling 150 GHz!). Each workstation comes with
1 GB memory.  Bla bla.

GH06: A common data directory {\tt /soft/gh06} is available on all 25
linux lab workstations. These machines are 
named {\tt c01tgh}, {\tt c02tgh} ... {\tt c24tgh} and the server
is called 
{\tt ctgh06}.   Each student has been assigned an account {\tt stud01}, 
{\tt stud02}, ... {\tt stud50}, the initial passwords are written
on the white-board, but please change them. Also note that depending
on your {\tt studXX} account, you will need to be on one of the
specific {\tt cYYtgh} machines. There is a simple algorithm between
{\tt XX} and {\tt YY}. See also the whiteboard.
% {\it should we put all students in a unix group, with default group write
% permission using umask 002??}

Printing should be done via the HP LaserJet 4050 in the lab. For those connecting their
laptops, use the ``Networked JerDirect'' queue type, with printer at
192.168.100.176 on port number 
9100\footnote{in CUPS this shows up as{\tt DeviceURI socket://192.168.100.176:9100}}

Here it is assumed that you are using 
the C-shell ({\tt csh} or {\tt tcsh}) or Bourne Shell  ({\tt bash}).
Easiest is to add the following  command to your {\tt .cshrc} file
for csh

\begin{verbatim}
    source /soft/gh06/manybody/manybody_start.csh
\end{verbatim}
or {\tt .bashrc} for bash users
\begin{verbatim}
    source /soft/gh06/manybody/manybody_start.sh
\end{verbatim}

This will add the NEMO and STARLAB packages to your environment, as well as
a number of other tools. It will also give you access to
a modest amount of source code that has not been compiled, and a number
of papers drawn from ADS and astro-ph and placed in {\tt \$MANYBODY/papers}.
If you want to install/copy this onto your laptop or home machine, please
see Appendix B.

\section{A quick test}

To verify the installation was done correct, here are some quick
one liners, without much of an explanation. Some of them will bring up
a picture. Most commands, except where noted, take a few seconds. If
you prefer, you can also execute the {\tt quick\_test} script in
the {\tt \$MANYBODY} root directory.
\begin{enumerate}

\item
{\tt firefox \$MANYBODY/index.html} : local HTML 

\item
{\tt mkplummer p1 1000} : NEMO, create a Plummer sphere

\item
{\tt snapplot p1 color=x} : NEMO, check if pgplot works. check color bands 

\item
{\tt gyrfalcON p1 . tstop=0.1} : NEMO, check if an integrator works

\item
{\tt mkkd95 a0.dat}: NEMO: create composite galaxy (takes 45 secs on a P1.6GHz)

\item
{\tt snapplot a0.dat color='i<8000?2.0/16.0:(i<12000?3.0/16.0:4.0/16.0)' yvar=z xrange=-8:8 yrange=-8:8} :
NEMO: parse bodytrans functions?

\item
{\tt snapxyz a0.dat - color='i<8000?1:(i<12000?2:4)' | xyzview - maxpoint=18000 nfast=18000 scale=8 fullscreen=t} : 
NEMO 3D viewer? Hold down key '1', '2' or '2' and move the mouse. Hit ESC to quit.

\item 
{\tt glnemo a0.dat select=0:7999,8000:11999,12000:17999} : 
NEMO 3D viewer. Press left mouse to view the object from different angles,
use the scroll mouse (if you have one) to zoom in and out.

\item 
{\tt snapgrid a0.dat - xrange=-8:8 yrange=-8:8 nx=128 ny=128 yvar=z | ccdsmooth - - 0.1 | ccdfits - a0.fits} :
NEMO make a CCD type observation of dark matter

{\tt ds9 a0.fits} : vizualize it. right mouse down changes contrast. Try zooming in and changing color map to SLS.

\item
{\tt nemobench gadget} : a small Gadget2 benchmark (TBA)

\item
{\tt makeplummer -n 32 > p32.dyn} : Starlab : create a Plummer sphere


\item
{\tt makeplummer -n 10 | dtos - | snapprint - } : a Starlab-NEMO pipe and converting data

\item
{\tt makeplummer -n 32 | kira -D 0.1 -t 2 | xstarplot} : Starlab w/ graphics. 'q' to quit.

\item
{\tt make -f \$MANYBODY/partiview/data/primbin16.mk} : Starlab: 16 body integrator. Takes about 15 secs.

\item
{\tt partiview  \$MANYBODY/partiview/data/primbin16.cf} : Partiview: view animation. Resize window.
Left mouse down rotates. Right mouse down zooms. $>>$ button starts movie.

\item
{\tt binsingle -D 1 | glstarview} : fewbody + visualizer

\item
{\tt g++ -o forward\_euler1 \$MANYBODY/acs1/chap3/forward\_euler1.C; forward\_euler1} 
(ACS 1)

\item
{\tt ...ruby...}
(ACS 2)


\end{enumerate}

\section{Available packages and commands}

\begin{verbatim}
  NEMO             package with 200+ tools
  Starlab          package with 200+ tools

  partiview        4D (space+time) data viewer
  ds9              a FITS image display program
  fewbody          programs that come with Fregeau's ``fewbody''
  nbody6           Aarseth's nbody6 integrator
  gadget2          SPH-treecode for cosmology
\end{verbatim}


%%%%%%%%%%%%%%%%%%%%%%%%%%%%%%%%%%%%%%%%%%%%%%%%%%%%%%%%%%%%%%%%%%%%%%%%%%%
% \cleardoublepage
% \part{Appendices}
\chapter                {manybody: N-body Compute Toolbox}



\section{Linux Cluster}

The directory {\tt /soft/gh06} on the 25 Linux Lab machines
contains all the {\it manybody} software (and then some)  
we are discussing at this school. We have
also made a DVD available, from which you should be able to reproduce this
toolbox on your own laptop or workstation at home. 
Some details on the installation of this are described in the next sections.

First an overview of the {\it manybody} hierarchy under {\tt /soft/gh06}:

\footnotesize
\begin{verbatim}
manybody/                       root directory
    opt/                        contains a {bin,lib,include,...} tree for a private --prefix
    acs/                        kali code
    nemo_cvs/                   NEMO package
    starlab_cvs/                Starlab package
    partiview_cvs/              partiview visualization 
    starcluster_cvs/            starlab interface to web based programs
    fftw-2.1.5/                 fftw library 
    papers/                     ADS and astro-ph papers (PDF and PS mostly)

manybody_pkg/                   (mostly) tar balls of codes 
    
\end{verbatim}
\normalsize



\section{The DVD}
In the root directory of your DVD you will find the following files and directories:

\begin{verbatim}
    README                a small intro
    install_fc5           shell script that should install binary "manybody"
    index.html            top level index to all HTML in this tree
    quick_test            shell script to test MANYBODY (see Appendix A.2)

    manybody_fc5.tar.gz   FC5 binaries of manybody, without the papers
    papers.tar.gz         astro-ph and ADS paper for the manybody material

    manybody_pkg/         directory with all sources in case of re-install
\end{verbatim}

\section{Installation}

If your linux system is not Fedora Core 5
or compatible, recompilation may be needed, since its shared library requests
may be conflicting with the ones your system has.
The script
{\tt install\_manybody} should guide you through this. You can find a copy
of this in either the {\tt manybody} directory, or its original CVS location,
{\tt \$NEMO/usr/manybody}.

\footnotesize\begin{verbatim}
  1% tar zxf /mnt/cdrom/manybody_fc5.tar.gz
  2% source manybody/manybody_start.csh
  3% tar zxf -C manybody /mnt/cdrom/papers.tar.gz

\end{verbatim}\normalsize

if you need to do a source install, the following should be 
your starting point\footnote{the root name {\tt /mnt/cdrom} may
very well be different on your machine}

\footnotesize\begin{verbatim}
 11% /mnt/cdrom/install_manybody pkg=/mnt/cdrom/manybody_pkg
 12% source manybody/manybody_start.csh
\end{verbatim}\normalsize

but more than likely some tweaking is needed if you have missing or
incompatible libraries or tools. 


\footnotesize\begin{verbatim}
   4% setenv CC  gcc-32
   5% setenv CXX g++-32
   6% setenv F77 g77
   7% /mnt/cdrom/install_manybody pkg=/mnt/cdrom/manybody_pkg
\end{verbatim}\normalsize


A typical install takes about an hour.

\section{Importing new Packages into manybody}

Much of the available open source software we use, use
{\it autotools} to {\tt configure}, {\tt make} and {\tt install} their
software on the users system. Where exactly this software is going to be located,
is free to the user to decide. On most Unix-based system the {\tt /usr},
{\tt /usr/local}, {\tt /opt/local} {e.g. DarwinTools on MacOSX} 
or {\tt /sw} (e.g. Fink on MacOSX) are used for this.
As long
as the user then puts the respective {\tt ''bin''} directory in their search
path, and perhaps the respective {\tt ''lib''}  directory in their 
shared library search path, all is well.

The drawback of this approach is that the user needs have administrative
privilages (username {\tt root} in Unix parlor). For {\bf manybody} we
cannot always assume that, so we choose a similar approach whereby libraries, programs
and their ancillary material are placed in a directory under user control.


So, for almost all such packages we can use these kinds of commands:

\footnotesize\begin{verbatim}

  1% tar zxf /tmp/package-X.Y.Z.tar.gz
  2% cd package-X.Y.Z
  3% ./configure --prefix=$MANYBODY/opt
  4% make
  5% make install

\end{verbatim}\normalsize

and for packages that depend on other packages, we would use something like

\footnotesize\begin{verbatim}

  3% ./configure --prefix=$MANYBODY/opt --with-fltk=$MANYBODY/opt

\end{verbatim}\normalsize


\section{Example install session}

\footnotesize\begin{verbatim}
% install_manybody
install_manybody : version ...
Wed Jul 5 14:56:00 EDT 2006 Using pkg=/scratch11/teuben/manybody_pkg
Wed Jul 5 14:56:01 EDT 2006 Installing cvsutils
Wed Jul 5 14:56:02 EDT 2006 Installing ruby
Wed Jul 5 14:57:18 EDT 2006 Installing acs
Wed Jul 5 14:57:22 EDT 2006 Installing starlab
Starlab version 4.4.2 loaded with
        STARLAB_PATH         = /scratch11/teuben/manybody/starlab_cvs
        STARLAB_INSTALL_PATH = /scratch11/teuben/manybody/starlab_cvs/usr
STARLAB-COUNT: 176
Wed Jul 5 15:05:09 EDT 2006 Installing starcluster
ERRORS compiling starcluster, check /scratch11/teuben/manybody/tmp/starcluster.log
ERRORS installing starcluster, check /scratch11/teuben/manybody/tmp/starcluster.log
STARLAB+CLUSTER-COUNT: 176
Wed Jul 5 15:05:09 EDT 2006 Installing fltk
Wed Jul 5 15:06:03 EDT 2006 Installing partiview
ERRORS compiling partiview, check /scratch11/teuben/manybody/tmp/partiview.log
cp: cannot stat `partiview': No such file or directory
Wed Jul 5 15:06:25 EDT 2006 Installing NEMO
NEMO-COUNT: 209
Wed Jul 5 15:10:51 EDT 2006 Installing gadget
Wed Jul 5 15:29:07 EDT 2006 Installing firstn
Wed Jul 5 15:29:08 EDT 2006 Installing nbody6
Wed Jul 5 15:29:29 EDT 2006 Installing galactICS
Wed Jul 5 15:29:34 EDT 2006 Installing gsl
Wed Jul 5 15:33:44 EDT 2006 Installing fewbody,glstarview
ERRORS compiling glstarview, check /scratch11/teuben/manybody/tmp/fewbody.log
cp: cannot stat `glstarview': No such file or directory
Wed Jul 5 15:33:47 EDT 2006 Installing hnbody- for linux only
Wed Jul 5 15:33:47 EDT 2006 Installing ds9 - for linux only
Wed Jul 5 15:33:47 EDT 2006 Installing xyz from NEMO
Wed Jul 5 15:33:55 EDT 2006 Installing EZ
ERROR: no known fortran compiler for EZ available (ifort, f95 g95)
Wed Jul 5 15:33:56 EDT 2006 Installing Tioga
Wed Jul 5 15:34:04 EDT 2006 Installing MMAS
Wed Jul 5 15:34:04 EDT 2006 Installing FFTW
Wed Jul 5 15:34:48 EDT 2006 Instaling StarCrash
Wed Jul 5 15:34:53 EDT 2006 Done.



\end{verbatim}\normalsize

In this particular example you see some errors occurred, {\tt glstarview} and
{\tt partiview} did not
properly build on this machine (a missing library), and there was no fortran-95
compiler available for EZ to be compiled. Even though NEMO returned with 209
binaries, this is not a full success, where 222 have been seen.

\section{ACS}

Say something about the Hut \& Makino {\it ACS} series. ACS1 (C++) vs. ACS2 (ruby).



%%%%%%%%%%%%%%%%%%%%%%%%%%%%%%%%%%%%%%%%%%%%%%%%%%%%%%%%%%%%%%%%%%%%%%%%%%%
%\cleardoublepage
\chapter                {Using CVS}

If you have never heard of CVS it is worth reading
this appendix and considering to use it (a few packages
in manybody are CVS enabled, encouraging you to use this timesaving mode).
CVS is one of the more popular
source code control systems, which simplifies keeping your source code
up to date with a master version. It can be useful for collaborators
to work on a project (source code, a paper), but also for a single
developer testing out code on various machines.

\section{Anonymous CVS}

Most likely you will first start by using the so-called ``anonymous CVS''
method, which works much like anonymous-ftp.  It is  however
important to enable your CVS account first, using the {\tt cvs login}
command\footnote{In this and age of security, there is a chance this
command will hang and not continue. Port 2401 may be blocked by a 
router near your connection}:

\footnotesize\begin{verbatim}
  1% cvs -d :pserver:anonymous@cvs.astro.umd.edu:/home/cvsroot login
  Logging in to :pserver:anonymous@cvs.astro.umd.edu:2401/home/cvsroot
  CVS password: 
\end{verbatim}\normalsize

simply hit return here, since there is no password. You only need to do
this once per CVS account, as the account information is added to a file
{\tt .cvspass} in your home directory.


\section{Starting from scratch}

\begin{enumerate}

\item The environment variable CVSROOT, or the -d flag to the cvs command,
is needed to get accesss to a repository. Use the one listed in the
previous section after the {\tt -d} flag.

\item You then need to checkout a new sandbox that mirrors a repository
module (the {\tt -Q} flag makes it much less verbose}:
\footnotesize\begin{verbatim}

     # checkout nemo, assuming CVSROOT has been set
  %1 cvs -Q co nemo

     # checkout starlab, notice the somewhat odd looking module name under manybody
  %2 cvs -Q co -d starlab manybody/starlab

\end{verbatim}\normalsize




\end{enumerate}


\section{Starting with an existing package that is CVS enabled}

If you already have a directory tree that is ``CVS enabled'' (each
directory will have a {\tt CVS} subdirectory in which administrative
details of that directory are stored), life is even easier.

\begin{enumerate}

\item By default the contents of the {\tt CVS/Root} file(s) will be used to
get access to the repository. Otherwise, as before, the CVSROOT environment
variable, or the -d command option flag, can be used. Basically you
don't need to worry about setting CVSROOT or using the -d flag here.

\item To check your source code for any needed updates: {\tt cvs -n -q update}.
You might get a response like:
\footnotesize\begin{verbatim}
  ? src/kernel/io/try.c            <-- file not under CVS control
  U man/man1/mkplummer.1           <-- newer file on the server
  M man/man5/data.5                <-- you have a modified file
  C src/nbody/init/mkplummer.c     <-- a conflict!!! both server and you have modified
\end{verbatim}\normalsize
The first column designates the status of the file. 
We will treat the listed 4 cases seperately (there are a few more, but not
so common):

\item[{\tt ?}] 
These files can be safely ignored. They happened to be in the directory
as a side-effect of installation or some other tinkering you did. They could
be added to CVS using the {\tt cvs add} command.

\item[{\tt U}] 
This means the file is new(er) at the server, and your
CVS sandbox needs to be updated. A simple command: {\tt cvs update}.

\item[{\tt P}] You might see this when you {\tt cvs update} a file, instead
of Updating the file, it is patched, taking much less bandwith.

\item[{\tt M}] This means your version of the file is newer than on
the server, 
and as long as you have write permission on the server, it can be returned
with a simple command: {\tt cvs commit}.

\item[{\tt C}] This is the more complicated case of a conflict. Both your local
version, as well as the server version, have been modified independently.
Most of the time CVS is actually able to make a version that merges both
modifications, though the developer should now check if the resulting code
is correct. The correct CVS order to fix this is a three step process: the
code is updated, then edited and checked for correctness, and finally
committed back to the repository:

\footnotesize\begin{verbatim}
  1% cvs update mkplummer.c
  2% make/edit/check/debug mkplummer.c
  3% cvs commit mkplummer.c
\end{verbatim}\normalsize


\end{enumerate}

\section{cvsutils}

In your {\it manybody} environment you will find a few perl scripts
(dubbed {\tt cvsutils})\footnote{See {\tt http://www.red-bean.com/cvsutils/}}
that help with some tedious CVS interactions, notably some tools when
you are not online. Another common enough operation is to change the 
{\tt CVS/Root}
(where your local {\tt \$CVSROOT} lives) from anonymous to somebody with
write permission, viz.

\footnotesize\begin{verbatim}
  1% cd $NEMO
  2% cat CVS/Root
  :pserver:anonymous@cvs.astro.umd.edu:/home/cvsroot
  3% cvschroot :pserver:pteuben@cvs.astro.umd.edu:/home/cvsroot
\end{verbatim}\normalsize

\section{svn}

The future of CVS is unclear. A lot of open source packages are  switching
to SubVersion (command: {\tt svn}). In practice the {\tt cvs} and 
{\tt svn} commands are nearly interchangeable.

\section{git}

A more recent innovation is the use of distributed CMS, such as {\tt git}. This
has the advantage of maintaining/creating your own repository and only later merge
them. The linux kernel is currently maintained using {\it git}.

\begin{figure}[htb]
\plotone{cvs.eps}
\caption[CVS diagram]
{CVS flow diagram}
\label{f:cvs}
\end{figure}



%%%%%%%%%%%%%%%%%%%%%%%%%%%%%%%%%%%%%%%%%%%%%%%%%%%%%%%%%%%%%%%%%%%%%%%%%%%
%\cleardoublepage
\chapter                {Using NEMO}

A summary of useful things to know about NEMO:
\begin{enumerate}

\item
NEMO is a large (250+) collection of programs, each performing a small task
controlled by a set of parameters.

\item
the command line user interface (CLI) is a set of {\it program keywords}
(unique to each program) and {\it system keywords}
(available to each program).
Most often used system keywords are {\tt yapp=} (plotting device),
{\tt debug=} (increase debug level) and {\tt error=} (pass over fatal errors
at your own risk).

\item
each program displays internal help using the option 
{\tt --help}, {\tt help=} or {\tt help=h}. Use \verb+help=\?+ to see
all the kinds of help the CLI can give. Use
the {\tt man} or {\tt gman} command to view or browse manual pages.

\item
command line parameters do not need to have their keyword name given
if the parameters are given in the right order. 
E.g. {\tt mkplummer p10 10} is the same as 
{\tt mkplummer out=p10 nbody=10}. Use {\tt help=} to see that order.

\item
most programs have a manual page. Use the unix {\tt man} command, or
GUI tools like {\tt gman}  [under redhat this is hidden in the yelp package]

\item
data is mostly stored in binary files, for speed, accuracy and portability.
Programs like {\tt tsf} display the contents of these files.

\item
in  Unix tradition data is often piped from one program to the next, the
{\tt in=/out=} filename  needs to be designated with a dash 
(e.g. {\tt in=-, out=-}.   Most programs have in/out as their 1st and 
2nd parameter.

\item
a large number of programs produce ASCII tables. Programs like
{\tt tabplot} and {\tt tabhist} are convenient endpoints to
quickly present results graphically.

\end{enumerate}

One or two overview plots on the packages and file formats in NEMO will now follow.


%%%%%%%%%%%%%%%%%%%%%%%%%%%%%%%%%%%%%%%%%%%%%%%%%%%%%%%%%%%%%%%%%%%%%%%%%%%
%\cleardoublepage
\chapter                {Using Starlab}

A summary of useful things to know about Starlab:


\begin{enumerate}

\item
Starlab is a large (250+) collection of programs, each performing a small task
controlled by a set of parameters.

\item
the command line user interface (CLI) is a set of flags

\item
each program displays internal help using the option {\tt --help}.
It actually does this by tracking down the source code
and displaying the relevant help section of the source code.

\item
data is mostly stored in special ascii files, with a hierarchical structure

\item
in Unix tradition data is piped from one program to the next, and unlike
in NEMO, unknown data is generally passed onwards unmodified.

\end{enumerate}

%%%%%%%%%%%%%%%%%%%%%%%%%%%%%%%%%%%%%%%%%%%%%%%%%%%%%%%%%%%%%%%%%%%%%%%%%%%
%\cleardoublepage
\chapter                {Using AMUSE}


%%%%%%%%%%%%%%%%%%%%%%%%%%%%%%%%%%%%%%%%%%%%%%%%%%%%%%%%%%%%%%%%%%%%%%%%%%%
%\cleardoublepage
\chapter                {Cover Art}

The figures on the cover page were made in the following way, printed out
and cut and pasted the classical way.

\section{Collisional Orbit}
\label{s:pyth} 

Although the {\it Figure8} orbit is perhaps very artistic, it is not very representative
for orbits in collisional systems, in fact, almost more appropriate
for collisionless systems!
The collisional orbit choosen was drawn from the Pythagoran problem:
3 stars with masses 3, 4 and 5, on a pythagoran triangle where the
long sides opposite their mass have a length proportional to 
their mass. Initially
all three stars are at rest. A file {\tt \$NEMODAT/pyth.dat} contains
the initial conditions:


% ok, this left figure is the wrong one
\begin{figure}[t]
\plottwo{pyth0.ps}{pyth1.ps}
\caption[Pythagorean problem]
{Initial conditions for the Pythagorean problem (left, but notice **???**but they are**
these are not the official Szebeheley \& Peters configuration we use
in the example)
and the orbit of the intermediate mass (M=4) particle starting
at (x,y) = (-2,-1).}
\end{figure}



\footnotesize\begin{verbatim}
   # print out the initial conditions
1% snapprint $NEMODAT/pyth.dat m,x,y,z,vx,vy,vz
m  x  y  z  vx vy vz
3  1  3  0  0  0  0
4 -2 -1  0  0  0  0
5  1 -1  0  0  0  0

   # integrate for a while
2% nbody00 $NEMODAT/pyth.dat pyth.out eta=0.01 deltat=0.01 tcrit=100 eps=0 
time = 0   steps = 0   energy = -12.8167 cpu =          0 min
time = 0.01   steps = 1   energy = -12.8167 cpu =          0 min
...
time = 99.98   steps = 33940   energy = -12.7912 cpu =     0.0115 min
time = 99.99   steps = 33945   energy = -12.7912 cpu =     0.0115 min
time = 100   steps = 33949   energy = -12.7912 cpu =     0.0115 min
Time spent in searching for next advancement: 0.1
Energy conservation: 0.0254786 / -12.8167 = -0.00199189
Time resets needed 5977 times / 10001 dumps

   # view Red=0 Green=1 Blue=2
   # notice that although initially blue/green pair up, red intervenes
   # and takes off with green heading up, leaving blue to wander south...
3% snapplot pyth.out trak=t xrange=-4:4 yrange=-4:4 color='i==0?2.0/16.0:i==1?3.0/16.0:4.0/16.0'

   # notice the final breakup occurs around time=82.4
4% snapprint pyth.out t,r | tabplot -
5% snapprint pyth.out t,r | tabplot - xmin=80 xmax=84

   # need to track down some unexpected dependancies of the results on compiler version
   # and options. this is of course somewhat disturbing.
   # all the trouble starts at that very close encounter around T=82

6% stoo pyth.out - 1 100000 | orbplot - xrange=-4:4 yrange=-4:4 
\end{verbatim}\normalsize

The 3-body problem is a chaotic problem. By perturbing the initial conditions
even a little bit, you can get drastically different results. This was shown
in another paper by Shebehely \& Peters (1967AJ.....72.1187S), where the
positions of the 3 bodies were all perturbed by of order 0.04, and this 
created a periodic solution.


\section{Collisionless Orbit}

Here we are taking orbits from the logarithmic potential as defined
by Binney \& Tremaine (1987) in eq. (2.54) and (3.77).

$$
   \Phi(x,y) = {1\over 2} v_0^2
                    \ln{ \left( r_c^2 + x^2  + y^2/q^2 \right) }
$$

The orbits in their Figure 3-7, our Figure~\ref{f:log}, are for
$q=0.9, R_c=0.14, v_0=0.07$.

\footnotesize\begin{verbatim}

   # Create an orbit, much like the one in Figure 3-7 of BT87
1% mkorbit orb1 y=0.1 vx=-1 etot=-0.337 potname=log potpars=0,0.07,0.14,0.9
Using pattern speed = 0
pos: 0.000000 0.100000 0.000000
vel: -1.000000 1.330308 0.000000
etot: -0.337000
lz=0.100000

2% orbint orb1 orb1.out nsteps=10000 dt=0.01 ndiag=100
0.000000 1.384859 -1.721859               -0.337 -0
1.000000 0.074342 -0.411342     -0.3369998357925 -4.87263e-07
...
98.970000 1.434173 -1.771178     -0.3370046579526 1.38218e-05
99.980000 0.016410 -0.353415     -0.3370045122688 1.33895e-05
Energy conservation: 1.33895e-05

3% orbplot orb1.out xrange=-0.8:0.8 yrange=-0.8:0.8

\end{verbatim}\normalsize

\begin{figure}[t]
\plottwo{log1.ps}{log2.ps}
\caption[Orbits in a Logarithmic potential]
{Orbits in a Logarithmic potential:
Initial conditions y=0.1 result in a box orbit (left), 
y=0.2 in a loop orbit (right). See also Figure 3-7 in BT87}
\label{f:log}
\end{figure}

\section{Barred Galaxy}

Here we produce an image of a simulated barred galaxy, overlayed with a contour diagram
of a smoothed distribution of stars. First a barred galaxy is created by
integrating an unstable exponential disk for a few rotation times.

\footnotesize\begin{verbatim}

   # Create an unstable disk
1% mkexpdisk disk.in 4096 

   # integrate for a few rotation times
2% gyrfalcON disk.in disk.out tstop=5 step=1
#    time      energy       -T/U     |L|       |v_cm|  build   force    step     accum
# ------------------------------------------------------------------------------------
 0          -0.5751588    0.492   0.42204    2.2e-10  0.01   0.04       0.05      0.05
 0.015625   -0.575151     0.49177 0.42204    5.8e-10  0      0.05       0.05      0.11
 0.03125    -0.5751344    0.49121 0.42204    7.9e-10  0      0.06       0.06      0.17
...
 4.9531     -0.5766627    0.50137 0.42187    1.3e-08  0.01   0.03       0.04     14.54
 4.9688     -0.5766588    0.50042 0.42186    1.3e-08  0      0.04       0.04     14.59
 4.9844     -0.5766436    0.49937 0.42186    1.4e-08  0.01   0.04       0.05     14.64
 5          -0.5766284    0.49822 0.42186    1.4e-08  0      0.04       0.04     14.68

   # check up on energy conservation of the code
3% snapdiagplot disk.out
6 diagnostic frames read
Worst fractional energy loss dE/E = (E_t-E_0)/E_0 = 0.00255499 at T = 5

   # nice composite diagram showing the initial, bar, and messed up state.
4% snapplot disk.out times=0,1,2,5 nxy=2,2 nxticks=3 nyticks=3

   # now combine a particle plot with a smooth contour diagram
5% snapgrid disk.out ccd1 times=2 nx=256 ny=256
6% ccdsmooth ccd1 ccd2 0.1
7% ccdplot ccd2 0.03,0.1,0.3,1,3,10      yapp=p1.ps/vps
8% snapplot disk.out times=2 psize=0.01  yapp=p2.ps/vps 
9% catpgps p1.ps p2.ps > p12.ps

10% snapgrid disk.out - times=2 nx=256 ny=256 zvar=vy mom=0 | ccdsmooth - ccd3.0s 0.1
11% snapgrid disk.out - times=2 nx=256 ny=256 zvar=vy mom=1 | ccdsmooth - ccd3.1s 0.1
12% ccdmath ccd3.1s,ccd3.0s ccd3.vel %1/%2
13% ccdplot ccd3.vel -0.9:0.9:0.3 blankval=0 yapp=p3.ps/vps
14% catpgps p2.ps p3.ps > p23.ps

   # or if gif is used, left as an exercise to the reader

10% ccdplot ccd2 0.03,0.1,0.3,1,3,10          yapp=p1.gif/gif
11% snapplot disk.out times=2 color=2.0/16.0  yapp=p2.gif/gif
12% See ImageMagick(1)

\end{verbatim}\normalsize

% ok, this left figure is the wrong one
\begin{figure}[t]
\plottwo{p12.ps}{p23.ps}
\caption[A Barred Galaxy]
{the making of a Barred Galaxy. On the left the particle distribution
is overlayed with a 0.1 beam smeared density distribution. On the right
the overlayed contours are that of the Y velocities. Notice the S shaped
contours in the bar region, due to the elliptical orbits along the bar.
}
\end{figure}

\section{Disk-Bulge-Halo galaxy}

A composite Disk-Bulge-Halo model in near equilibrium can be created using the Kuijken-Dubinski
prescription. A NEMO frontend to their tools is available via the program {\tt mkkd95}. By
default it will create a disk with 8000 particles, a bulge with 4000 particles and a halo
with 6000 particles. {\tt snapplot} can be told to color particles (assuming it has been
compiled with a graphics driver that knows color, e.g. pgplot)

\footnotesize\begin{verbatim}

 1% mkkd95 a0.dat
 2% snapplot a0.dat color='i<8000?2.0/16.0:(i<12000?3.0/16.0:4.0/16.0)' yvar=z xrange=-8:8 yrange=-8:8 yapp=/gif

\end{verbatim}\normalsize

This particular plot actually has a black background, so after replacing black with white (e.g. using
Gimp or ImageMagic) and some axis cropping, the cover version is ready.

\section{Millenium Simulation}

Here we had to cheat for moment. This picture was piched from the Millenium simulation, where
Volker Springel deserves all the credit!   The cover picture is a small section sized
125 Mpc/h in the horizontal direction.


%%%%%%%%%%%%%%%%%%%%%%%%%%%%%%%%%%%%%%%%%%%%%%%%%%%%%%%%%%%%%%%%%%%%%%%%%%%
%\cleardoublepage
\chapter                {NEMO programming}

NEMO is written in vanilla (mostly ANSI) C, with moderate use of
macros. Most of this is written up in the 
{\it NEMO Users and Programmers Guide}. We'll just highlight a few 
things you might encounter if you would browse the source code:

\begin{enumerate}

\item
nemo\_main(); stdinc.h;

\item
Instead of the laborious work of opening files and error checking

\footnotesize\begin{verbatim}
    FILE *fp = fopen(argv[1],"r");
    if (fp==NULL) {
      fprintf(stderr,"File %s cannot be opened\n",argv[1]);
      exit(1);
    }
\end{verbatim}\normalsize

we typically use one-liners
\footnotesize\begin{verbatim}

    stream istr = stropen(getparam("in"),"r");
\end{verbatim}\normalsize
which cause a fatal error 
in those common cases of ``not found'', ``no permission'' etc.etc.
{\tt stropen} also takes care of special files such as {tt ``-''} and {\tt ``.''}.

\item
Instead of different types of {\tt printf()} statements, use
\footnotesize\begin{verbatim}
    error("Bad value of n=%d",n);
    warning("Fixing n=%d",n);
    dprintf(1,"Iteration n=%d sum=%g",n,sum);
\end{verbatim}\normalsize
controlled with the {\tt error=} and {\tt debug=} system keywords.

\item
Random number are normally initialized in a generic way, as follows:
\footnotesize\begin{verbatim}
    iseed = set_xrandom(getparam("seed"))

with the following meaning for seed:

    -2   pid
    -1   centi-seconds since boot
    0    seconds since 1970.0
    n>0  your personal seed selection
\end{verbatim}\normalsize

there are controlled with the  {\tt seed=} program keyword.


\end{enumerate}

\section{Creating a new program}

There are roughly three ways one can decide on how to add functionality
to NEMO (i.e. creating a new program):

\begin{enumerate}
\item  Add a keyword to an existing program with nearly the same functionality.
Be aware that you should
then make it backwards compatible, i.e. the default value for the keyword
should reflect the way it was working before to prevent surprising the
old-timers.  Also, don't forget to add
this new keyword to the manual page (NEMO/man/man1 typically)

\item
Clone an existing program, and give it a nice new name. This means you'll also 
have to clone the manual page (NEMO/man/man1 typically), 
and perhaps cross-references (in the SEE ALSO section) of a few other manual
pages. In most cases you may have to add an entry to the Makefile,
at least in the BINFILES= macro, to ensure it gets installed.
Sometimes special rules are needed for binaries.
If you want the program to be part of the standard distribution, add
an entry to the {\tt Testfile}, so it gets built when NEMO is installed
and runs the testsuite.
Also test, from {\tt \$NEMO}, if {\tt mknemo}
does not get confused about naming conventions before the program
is committed to CVS.

\item
Write a whole new program. You can either use {\tt NEMO/src/scripts/template},
and fill in the blanks, or write it from scratch. See also 
previous comments on Makefile, Testfile and mknemo.

\footnotesize\begin{verbatim}
    1% $NEMO/src/scripts/template foo a=1 b=2 n=10
    2% bake foo
    3% foo c=1 
    ### Fatal error [foo]: Parameter "c" unknown
\end{verbatim}\normalsize

\end{enumerate}

%%%%%%%%%%%%%%%%%%%%%%%%%%%%%%%%%%%%%%%%%%%%%%%%%%%%%%%%%%%%%%%%%%%%%%%%%%%
%\cleardoublepage
\chapter                {Maintenance and Updates}

Updates to the code can either be done via releases (tar files or
tagged CVS) or in a ``continuum'' (CVS).

\section{NEMO (version 3.2.3)}
After installation, it can occur that it appears 
as if you have a missing program in NEMO, despite that a 
manual page exists.

There are 3 quick and easy ways to install/update a program

\begin{enumerate}

\item
The program just needs to be compiled, or you modified it,and needs to be
placed back in the system:
{\tt mknemo program}

\item
The program was updated at the master site, so you need to update your CVS
repository, and re-compile it:
{\tt mknemo -u program}

\item
A number of library routines were updated, and therefore the program
should be recompiled and/or relinked:
{\tt mknemo -l -u program}

\end{enumerate}

\section{STARLAB (version 4.2.2)}

{\it be sure to update this: since the last summerschool starlab install
was overhauled}

Similar to NEMO there is a script {\tt mkstarlab} available which
makes it relatively easy to update or recompile the code under
most circumstances.

\begin{enumerate}

\item
Just (re)compile the program:
{\tt mkstarlab program}

\item
Fetch a new version via CVS and recompile a program:
{\tt mkstarlab -u program}

\item
Update the code, recompile the library and the recompile the program:
{\tt mkstarlab -l -u program}

\end{enumerate}

\section{Other}

...not yet...

%%%%%%%%%%%%%%%%%%%%%%%%%%%%%%%%%%%%%%%%%%%%%%%%%%%%%%%%%%%%%%%%%%%%%%%%%%%
%\cleardoublepage
\chapter                {Libraries}


%%%%%%%%%%%%%%%%%%%%%%%%%%%%%%%%%%%%%%%%%%%%%%%%%%%%%%%%%%%%%%%%%%%%%%%%%%%
%\cleardoublepage
\chapter                {Troubleshooting}

Compiling {\tt manybody} can be a chore, since it contains many packages
and modules, each of which can have some dependancies on the system.
If you are running linux, one of the most common things not installed
is the development libraries for X windows and perhaps a few other libraries.
For RH9 this would be {\tt  XFree86-devel}, for FC4 {\tt xorg-devel}.
Mesa is another often missed package (needed for partiview).

\section{Known Problems/ToDo's}
\begin{enumerate}

\item
The default shell should be (t)csh, there is some limited (ba)sh support. 

\item
nbody4 is now available, though may be called Brute4 (Nbody4 needs the
GRAPE hardware and Grape libraries)

\item
{\tt kira --help} sometimes will complain ``source file unavailable''

\item
The {\tt gman} command could be be missing on your machine. 
It's very nice to have though!

\item
Q=0 for runbody1 when in= is used...   bug in the wrapper interface

\item
{\tt dtos} suffers from some quick hacks put in for the AAS NVO demo:
Aux/Acceleration/Potential are meaningless.

\item
{\tt snapplot} multipanel has odd colors (grey) showing up

\item
fix MANPATH for NEMO under linux

\item
{\tt tkrun} demos / python?

\item
overview diagrams of the hierarchy and layout of NEMO, Starlab, Manybody

\item
add Kawaii's code

\item
units(1) and units(5) unfinished

\item
(i)python, matplotlib and various interfaces for NEMO?

\item
Fink/DarwinPorts

\item
NAM (Shaya/Peel)

\item
Romeo's wavelet code (has some GPL issues)

\item 
SuperBox

\item
starlab has some installation
problems some compilers (e.g. on FC5, but mdk10 ok) - related to their internal
autoconf version?

\item
Greengard \& Rochlin's FMM Multipole Moment Code (simple C version via Umiacs)

\item
single vs. double precision:  gadget can actually be compiled in double precision
mode, but the gadgetsnap/snapgadget programs are likely to break.  gyrfalcON is
mostly compiled in double mode in NEMO, but Walter will most likely
prefer float.

\item 
(gyr)falcON may need to be re-compiled without the  {\tt -ffast-math} option
(see {\tt OPTFLAGS} in  {\tt make/defs}). Otherwise it has the tendency to crash.
(e.g. AMD64 with gcc 3.4.5 and intel with gcc 4.1.0)

\item 
intel compiler can speed up a lot 

\item
64 bit issues:
one of the current problems with fortran is the switch from g77 to gfortran. The former
cannot handle many 64 bit issues, where gfortran is able to do so. gfortran is however
more ANSI strict in language features, but does handle (some) fortran-95 . To make 
things more confuising, your system may also contain {\tt g95}. If you decide to compile
NEMO with gfortran, make sure no g77 is used where data is transformed with {\tt unfio},
as the header size is 4 in g77. There are also compiler issues. Sometimes
gfortran is faster (magalie improved from 49\" to 22\", but mkkd95 ran slower,
from 38\" to 56\"!).

\end{enumerate}

\section{Unknown Problems}

This section merely exists to not confuse Piet, and is otherwise
left to your investigation.

\end{document}

%%%%%%%%%%%%%%%%%%%%%%%%%%%%%%%%%%%%%%%%%%%%%%%%%%%%%%%%%%%%%%%%%%%%%%%%%%%
\cleardoublepage
\chapter*{cover art figures}
\begin{figure}[t]
\plottwo{pyth1.ps}{}
\end{figure}

\begin{figure}[b]
\plottwo{log1.ps}{}
\end{figure}

\begin{figure}[b]
\plottwo{p23.ps}{}
\end{figure}







%  other random stuff



Fellhauer, M.; Kroupa, P.; Baumgardt, H.; Bien, R.; Boily, C. M.; Spurzem, R.; Wassmer, N.
2000NewA....5..305F
SUPERBOX - an efficient code for collisionless galactic dynamics
