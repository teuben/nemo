%       30-aug-93  Created                          Peter Teuben
%%%%%%%%%%%%%%%%%%%%%%%%%%%%%%%%%%%%%%%%%%%%%%%%%%%%%%%%%%%%%%%%%%%%%%%%%%%%%%

\documentstyle[nemo]{report}

\title{NEMO PROGRAMMERS GUIDELINES}

\date{Version 2.1 \\
 August 1993 \\
 Document revised: \today\ by P. Teuben}

%%%%%%%%%%%%%%%%%%%%%%%%%%%%%%%%%%%%%%%%%%%%%%%%%%%%%%%%%%%%%%%%%%%%%%%%%%%
\begin{document}
\setlength{\parindent}{0pt}
\setlength{\parskip}{2.5mm}

\section*{GUIDELINES - SUMMARY}

\centerline{\underline{Programmers Guidelines - source code}}

If you modify existing NEMO source code, please consider the
following checklist:

\begin{itemize}
\item 

add/update a history trail with the modifications done (and perhaps
why) to this file. If you clone a program, don't copy the old trail,
say that you cloned it from program XXX on this day. For example:

\footnotesize\begin{verbatim}
/*
 *  History:   12-aug-93    V0.1    Cloned from snapplot.c              PJT
 *             30-aug-93    V1.0    added times= and select=, released  PJT
 *
 */
\end{verbatim}\normalsize

\item 
    
The {\tt VERSION=} keyword (always the last keyword) in the
{\tt string defv[]} array should under most circumstances be updated.
This is because it will appear in the datastreams created by  your program,
and at later times help anybody recover from reported bugs in programs.
Use {\tt 0.x} for alpha and beta-versions, and start with {\tt 1.0}
in a formal release. You can also use {\tt 1.4b} etc. for really
minor fixes. Major modifications, e.g. when keywords are added,
or their defaults have changed, you should change the major version
number, e.g. {\tt 2.0}.

\item  

Do you have a proper description in the now expected
{\tt string usage="..."} variable? If not, your program will now
inherite a vague description (a not so nice one)
from {\tt \$NEMO/src/kernel/io/usage.c}.

\end{itemize}

\centerline{\underline{Programmers Guidelines - manual page}}

Although programs can be submitted without a manual page, they will
not be accepted anymore in the {\tt \$NEMO/src/} without one!

\begin{itemize}
\item

If no man page is present, you can 
\begin{enumerate}

\item
create a template with ``{\tt mkman program > program.1}'' and edit
that template. Note that your {\tt program} must have been compiled
already for this to work.

\item
copy an existing one from {\tt \$NEMO/man/man1}
and edit it.

\end{enumerate}


\item
Append to the {\tt HISTORY UPDATE} at the end of the man page.

\item
Are all keywords up to date, do the defaults agree with those
in the program? They can be easily out of sync, since they
are in different files.

\item
Don't be shy about referencing other related manual pages in the
NEMO (or unix) manual tree, since this gives great power
to interactive hypertext man readers (such as {\tt hman} and
{\tt tkman}).

\end{itemize}

\centerline{\underline{Programmers Guidelines - Makefile}}

Note all programs need a makefile, but many can be compiled with
the template one.

\begin{itemize}
\item
Is {\tt bake} (i.e. {\tt make -f \$NEMOLIB/Makefile})  enough, or
do you realy need a special makefile? Multi-source programs
almost always need a special makefile.

\item
If you do need a special makefile, try and use macros that
are automatically imported to make. (NEMOLIB, YAPPLIB etc.)

\end{itemize}

\centerline{\underline{Converting source code to NEMO?}}

For many more details Chapter 6 in the {\tt NEMO Users Guide} will
tell you most of what you need to know, but here is a short
summary of the modifications needed to modify existing C programs
to NEMO.

\begin{itemize}

\item

Change the standard {\tt \#include <stdio.h>} to 
{\tt \#include <nemo.h>}\footnote{formally
{\tt stdinc.h} and, if the user interface is used,
{\tt getparam.h}}

\item

Change {\tt main(argc, argv)} to
{\tt nemo\_main()} (while linking, various proper start and finish
code will be executed by a {\tt main(argc,argv)} supplied by the
NEMO library).

\item

If you need values from the command line, use
{\tt getparam, getdparam, getiparam, getbparam} calls to obtain
resp. string, double, integer and boolean single valued keywords.
For arrays, or mixed types, more advanced routines, like
the {\tt nemoinpX} routines, are needed. They are beyond the scope
of this document.


\item

Replace {\tt fopen()} by {\tt stropen}, and 
{\tt fclose()} by {\tt strclose}, and 
This will not only give you
pipes, but also forces files in create-mode to be absent. 
Otherwise a fatal error message is given and the program
will exit.

\item

If you use NEMO's {\it filestruct(3NEMO)} in output mode,
be sure not to use {\tt printf} since it will interfere and
confuse the next program in a pipe. Replace it with
the equivalent {\tt dprintf(int debug\_level,...}, where
{\tt debug\_level} is a number 0 or larger. 
This output goes to {\it stderr}.

\item

Warning and fatal error messages should be issued with
{\tt warning(char *format, ...)}
and
{\tt error(char *format, ...)}. They operate much like
{\it printf(3)} and use the standard {\it varargs}
mechanism. The format string need not have a terminating
newline, it is automatically supplied.
This output also goes to {\it stderr}.

\item

If you want, you can change some types, e.g. {\tt char *} to
{\tt string} (unless they are used as pointers), {\tt real}
instead of {\tt double/float}, {\tt bool} instead of
{\tt int} (where appropriate). All this is defined in
{\tt stdinc.h}.

\item

If you can, device some standard kind of benchmark, of which
you know the answer (and keeping times along, this is also a
nice way to see improvements as hardware gets better). If you
do, comment it is the manual page how to perform the benchmark,
and perhaps even place a benchmark script in {\tt \$NEMO/csh}.
See also Appendix E in the {\it NEMO Users Guide}.

\end{itemize}

\footnotesize
\verbatimfile{example.src}
\normalsize

\end{document}


